\section{Funkcje wielu zmiennych}

Na początek kilka definicji dotyczących zbiorów w $ \mathbb{R}^n $.

\begin{itemize}
    \item \underline{Otoczenie} punktu $ P = (x_1, x_2, ..., x_n) \in \mathbb{R}^n $ to $n$ -- wymiarowa kula otwarta o środku
    w $P$ i promieniu $r > 0$, tzn. zbiór
    $$ K(P, r) = \{ Q = (y_1, y_2, ..., y_n) \in \mathbb{R}^n : |PQ|^2 = (x_1 - y_1)^2 + (x_2 - y_2)^2 + ... + (x_n - y_n)^2 < r^2 \} $$

    Dla $n=2$ jest to koło o środku w $P$ bez brzegowego okręgu.

    Dla $n=3$ jest to kula o środku w $P$ bez brzegowej sfery.
    \item \underline{Sąsiedztwo} punktu $x_0 \in \mathbb{R}^n$ to zbiór postaci $ S = S(P, r) = K(P,r) \backslash P $
    \item Zbiór $ A \subset \mathbb{R}^n $ jest zbiorem \underline{otwartym}, gdy każdy punkt z $A$ posiada pewne otoczenie zawarte w $A$, tzn.
    $$ \forall P \!\in\! A \ \exists K(P, r) \quad P\in K(P,r) \subset A $$
    \item Zbiór $ A \subset \mathbb{R}^n $ jest zbiorem \underline{domkniętym}, gdy jego dopełnienie $ A = \mathbb{R}^n \backslash A $ jest zbiorem otwartym. \\
\end{itemize}

\textbf{Definicja}

Funkcja wielu zmiennych ma postać
$$ f: D \to \mathbb{R} $$
gdzie $ D \subset \mathbb{R}^n $ jest dziedziną $f$.

Zatem dla $ (x_1, x_2, ..., x_n) \in D $ \quad $ f(x_1, x_2, ..., x_n) \in \mathbb{R} $

Gdy mamy funkcje dwóch zmiennych to zwykle piszemy $ z = f(x, y) $ a dla trzech zmiennych $ t = f(x, y, z) $.

Będziemy analizować głównie funkcje dwóch zmiennych $ z = f(x, y) $.

Dla takich funkcji można narysować wykres -- gdy $D$ jest otwarty to wykresem jest powierzchnia w $3$ wymiarach dana wzorem
$ (x, y, f(x,y)) $, gdzie $ (x,y) \in D_f $.

\begin{center}
\includegraphics[scale=0.6]{img/wykres_wieluzm1.png}

Wykres funkcji $ f(x,y) = x^3 + 3xy^2 - 51x - 24y, \quad -5 \leq x \leq 5, \ -5 \leq y \leq 5 $
\end{center}

\textbf{Wykresy niektórych popularnych funkcji}

\begin{itemize}
    \item $ z = Ax + By + C $ -- płaszczyzna o wektorze normalnym $ \vec{n} = [A, B, -1] $ i przechodząca przez punkt $ (0,0,C) $.
    \item $ z = z_0 + \sqrt{r^2 - (x-x_0)^2 - (y-y_0)^2} $ -- górna półsfera o środku w $(x_0, y_0, z_0)$ i promieniu $r > 0$. Np.
    $ z = 3 + \sqrt{7 - x^2 - (y-1)^2} : \ S(0,1,3), \ r=\sqrt{7} $

    $ z = z_0 - \sqrt{r^2 - (x-x_0)^2 - (y-y_0)^2} $ -- analogiczna półsfera ale dolna.

    Obie pochodzą z równania całej sfery: $ (x - x_0)^2 + (y - y_0)^2 + (z - z_0)^2 = r^2 $.
    \item $ z = z_0 + a\sqrt{(x-x_0)^2 + (y-y_0)^2}, \ a \neq 0 $ -- powierzchnia stożkowa o wierzchołku w $ P = (x_0,y_0,z_0) $
    i osi symetrii równoległej do osi $Z$.

    $a > 0$ -- wierzchołek w dół, $a < 0$ -- wierzchołek w górę.

    $|a| = \tan \alpha $, gdzie $\alpha$ jest kątem między prostą będącą tworzącą stożka

    \includegraphics[scale=0.5]{img/stozek_wlzm.png}
\end{itemize}

Powierzchnia stożkowa i półsfera są szczególnymi przypadkami tzw. \underline{powierzchni obrotowych} w $\mathbb{R}^3$. \\

Powierzchnią obrotową w $\mathbb{R}^3$ wokół osi $Z$ będziemy nazywali zbiór wszystkich możliwych punktów
$(x,y,z)$ taki, że podstawienie $r = \sqrt{x^2 + y^2}$ wyznacza zbiór $z$ jako współrzędne wszystkich par $(z,r)$
tworzących pewną krzywą na płaszczyźnie, przy czym zbiór wszystkich $r \geq 0$ jest zbiorem otwartym.

Zatem jeżeli ta powierzchnia jest dana przez pewne równanie postaci
$$ F(x,y,z) = 0 $$
to podstawienie $ r = \sqrt{x^2 + y^2} $ usuwa wszystkie $x$ i $y$ i prowadzi do równania zależnego tylko od $z$ oraz $r$.

W szczególności gdy mamy $z = f(x,y)$ i podstawienie $r$ powoduje, że $f$ zależy tylko od $r$ to wykresem $f$ jest powierzchnia
obrotowa wokół osi $Z$. \\

\textbf{Geometryczne własności takiej powierzchni}

\begin{itemize}
    \item Niepuste przecięcie powierzchni z dowolną płaszczyzną prostopadłą do osi $Z$ jest punktem, okręgiem lub sumą tych zbiorów.
    \item Niepuste przecięcie powierzchni z dowolną płaszczyzną zawierającą oś $Z$ jest krzywą o tym samym kształcie. \\
\end{itemize}

Na przykład dla powierzchni stożkowej $ z = a\sqrt{x^2 + y^2}, \ a > 0 $, przecięcie płaszczyzną prostopadłą do osi $Z$ jest okręgiem
lub wierzchołkiem, a przecięcie płaszczyzną zawierającą oś $Z$ jest sumą dwóch półprostych wychodzących z wierzchołka.

\begin{center}
\includegraphics[scale=0.6]{img/stozek_wlzm_2.png}
\end{center}

Sposób rysowania takich powierzchni opiera się na spotstrzeżeniu, że dla $x = 0$ i $y \geq 0$ mamy $ r = \sqrt{y^2} = y \geq 0 $.
Zatem rysujemy w płaszczyźnie $YZ$ wykres odpowiedniej krzywej dla $y \geq 0$, a następnie obracamy go wokół osi $Z$. Tworzy to
żądaną powierzchnię obrotową. \\

Poprzedni przykład raz jeszcze: $ z = a\sqrt{x^2 + y^2}, \ a > 0 $.

Tutaj dla $ r = \sqrt{x^2 + y^2} \geq 0 $ mamy $z =ar$. Zatem biorąc $ r = y \geq 0 $ w płaszczyźnie $YZ$ dostajemy wykres
funkcji liniowej $ z = f(0, y) = ay, \ y \geq 0 $. Jest to półprosta

\begin{center}
\includegraphics[scale=0.6]{img/stozek_wlzm_3.png}
\end{center}

Rozszerzanie powyższego przypadku -- powierzchnia obrotowa wokół osi równoległej do osi $Z$.

Jeżeli dla pewnych $x_0, y_0 \in \mathbb{R}$ podstawienie $ r = \sqrt{(x - x_0)^2 + (y - y_0)^2} $ usuwa wszystkie $x$ i $y$ i prowadzi
do równania zależnego tylko od $z$ oraz $r$ to dana powierzchnia jest powierzchnią obrotową wokół prostej
$ L : x = x_0, \ y = y_0, \ z\in \mathbb{R} $.

Jest to zatem przypadek powierczhni opisanej poprzednio (czyli dla $x_0 = y_0 = 0$) ale przesunięty następnie o wektor $ \vec{v} = [x_0, y_0, 0] $. \\

\textbf{Przykład}

Powierzchnia dana równaniem $ z = (x+2)^2 + (y-1)^2 $
Tutaj mamy $ x_0 = -2 $ oraz $ y_0 = 1 $ i podstawienie $ r = \sqrt{(x+2)^2 + (y-1)^2} $ daje równanie $ z = r^2, \ r \geq 0 $.
Zatem biorąc $ r = y \geq 0 $ w płaszczyźnie $YZ$ dostajemy wykres funkcji $ z = f(0, y) = y^2, \ y \geq 0 $. Jest to prawa gałąź
paraboli.

Obracając ją następnie wokół osi $Z$ dostajemy powierzchnię zwaną paraboloidą.

\begin{center}
\includegraphics[scale=0.6]{img/paraboloida.png}
\end{center}

Na koniec przesuwamy powyższą powierzchnię o wektor $ \vec{v} = [x_0, y_0, 0] = [-2, 1, 0] $ i to daje naszą powierzchnię.

\subsection*{Inny typ powierzchni -- tzw. powierzchnie walcowe}
\addcontentsline{toc}{subsection}{Powierzchnie walcowe}

Powierzchnia jest nazywana powierzchnią walcową równoległą do osi $Z$ jeżeli z faktu, że punkt $ (x_0,y_0,z_0) $
należy do powierzchni wynika, że dla dowolnego $z$ każdy punkt postaci $(x_0, y_0, z_0)$ też należy do tej powierzchni.

To oznacza, że jeżeli taka powierzchnia jest dana przez pewne wyrażenie to równanie to \textbf{nie zawiera} zmiennej $z$.

Geometrycznie -- niepuste przecięcie powierzchni z dowolną płaszczyzną równoległą do osi $Z$ daje krzywą o tym samym kształcie.

Stąd sposób tworzenia wykresów takich powierzchni -- rysujemy w płaszczyźnie $XY$ (czyli dla $z=0$) krzywą zadaną wyjściową relacją,
a potem wykres tej krzywej przesuwamy wzdłuż osi $Z$ i to generuje daną powierzchnię. \\ 

Dwa pozostałe przypadki są analogiczne:

\begin{itemize}
    \item gdy relacja definiująca powierzchnię nie zawiera $x$ to rysujemy odpowiednią krzywą w płaszczyźnie $YZ$
    , a potem jej wykres przesuwamy wzdłuż osi $X$,
    \item gdy relacja definiująca powierzchnię nie zawiera $y$ to rysujemy odpowiednią krzywą w płaszczyźnie $XZ$,
    a potem jej wykres przesuwamy wzdłuż osi $Y$.
\end{itemize}

Stąd prosta reguła -- odpowiednią krzywą przesuwamy zawsze wzdłuż tej osi, która odpowiada zmiennej \textbf{nieobecnej} w równaniu. \\

\textbf{Przykład}

Powierzchnia o równaniu $ x^2 + y^2 = 1 $.

Nie występuje $z$, a więc jest to powierzchnia walcowa równoległa do osi $Z$.

Wyznaczamy krzywą daną powyższą relacją w płaszczyźnie $XY$ -- jest to okrąg o środku w układzie współrzędnych i promieniu równym $1$.

Po przesunięciu tego okręgu wzdłuż osi $Z$ zostaje wygenerowana powierzchnia -- jest to powierzchnia boczna walca o nieskończonej długości.
Stąd bierze się nazwa tego typu krzywych.

\begin{center}
\includegraphics[scale=0.6]{img/walec.png}
\end{center}

\textbf{Definicja}

Poziomica funkcji $ z = f(x,y) $ na wysokości $h$ to zbiór
$$ D_h = \{ (x,y): f(x,y) = h \} $$
Jest to rzut na płaszczyznę $XY$ zbioru -- najczęściej krzywej -- będącego przekrojem wykresu $f$ płaszczyzną o równaniu $z = h$. \\

\textbf{Interpretacja geograficzna}

Jeśli płaszczyzna $XY$ jest "mapą" i wyznacza "poziom morza", $z$ -- wysokością nad "poziomem morza", a wykres $f$ jest "rzeźbą terenu"
to poziomica jest krzywą na "mapie" która łączy punkty odpowiadające tej samej "wysokości" $h$.

Na podstawie zagęszczenia poziomic dla odpowiednio dobranych $h$ możemy przewidzieć kształt wykresu $f$ -- czy jest stromy czy płaski.


\subsection*{Pochodne cząstkowe pierwszego rzędu funkcji wielu zmiennych}
\addcontentsline{toc}{subsection}{Pochodne cząstkowe pierwszego rzędu funkcji wielu zmiennych}

Są to pochodne danej funkcji liczone względem jednej zmiennej, a pozostałe zmienne są stałe i przyjmują rolę parametrów.

Oznaczenie dla $ f = f(x,y) $:

$$ \dpartial{f}{x} \ \textrm{lub} \ f_x \ \textrm{-- pochodna po} \ x $$
$$ \dpartial{f}{y} \ \textrm{lub} \ f_y \ \textrm{-- pochodna po} \ y $$

Formalna definicja: 
$$ \dpartial{f}{x} (x_0,y_0) = \lim_{h \to 0} \frac{f(x_0 + h, y_0) - f(x_0, y_0)}{h} $$

$$ \dpartial{f}{y} (x_0,y_0) = \lim_{h \to 0} \frac{f(x_0, y_0 + h) - f(x_0, y_0)}{h} $$

Dla funkcji $n$ zmiennych $ f = f(x_1, x_2, ..., x_n) $:
$$ \dpartial{f}{x_i} (x_1, x_2, ..., x_n) = \lim_{h \to 0} \frac{f(x_1, x_2, ..., x_{i - 1}, 
\textcolor{blue}{x_i + h}, x_{i + 1}, ..., x_n) - f(x_1, x_2, ..., \textcolor{blue}{x_i}, ..., x_n)}{h} $$ \\


\subsection*{Interpretacja geometryczna dla funkcji 2 zmiennych}
\addcontentsline{toc}{subsection}{Interpretacja geometryczna dla funkcji 2 zmiennych}

Wykres każdej funkcji $f$ dwóch zmiennych można przeciąć płaszczyzną równoległą do osi $Z$. Powstaje wtedy pewna krzywa, która jest częścią wspólną wykresu $f$
oraz płaszczyzny. Jest to szczególny przypadek tzw. funkcji \underline{warunkowej} o której wkrótce powiemy więcej. \\

Gdy taka krzywa jest regularna to możemy liczyć dla niej pochodną.

Gdy płaszczyzna przekroju przechodzi przez punkt $ P=(x_0, y_0, f(x_0, y_0)) $ to pochodna tej krzywej jest równa

\begin{itemize}
    \item $ \dpartial{f}{x} (x_0, y_0) $, gdy płaszczyzna jest $\parallel XZ $,
    \item $ \dpartial{f}{y} (x_0, y_0) $, gdy płaszczyzna jest $\parallel YZ $. 
\end{itemize}

\begin{center}
\includegraphics[scale=0.4]{img/interpretacja_geom.png}
\end{center}

\textbf{Sposób wyznaczania pochodnych cząstkowych w praktyce}

Ponieważ tylko jedna zmienna jest w użyciu, a pozostałe stają się parametrami to korzystamy z reguł różniczkowania
funkcji $1$ zmiennej.

Pamiętać należy, że dla wybranej zmiennej dowolne wyrażenie z każdą inną zmienną \textbf{staje się stałą} i jej pochodna
po wybranej zmiennej jest \textbf{równa 0}.

Czyli np.
$$ \dpartial{}{y} (4x^2 + 3 \sin x + 5) = 0, \quad \dpartial{}{x} (ye^{z + 2y}) = 0 \quad \textrm{itd.} $$ \\

\textbf{Przykład}

$$ f(x,y) = x \sin (xy^3) $$
Wtedy różniczkując po $x$ mamy pochodną iloczynu:
$$ \dpartial{f}{x} = f_x = ((x)_x \cdot \sin(xy^3))_y = x \cdot (\sin(xy^3))_y = x \cdot \cos(xy^3) \cdot 3y^2 x$$ \\


\subsection*{Pochodne drugiego rzędu}
\addcontentsline{toc}{subsection}{Pochodne drugiego rzędu}

Mając pochodne $1$ rzędu definiujemy pochodne drugiego rzędu jako pochodne pierwszego rzędu z pochodnych pierwszego rzędu.
W szczególności, dla $ f = f(x,y) $ mamy $4$ pochodne drugiego rzędu. \\

Pochodne \underline{jednorodne} po danej zmiennej:
\begin{itemize}
    \item $ \frac{\partial^2 f}{\partial x^2} = \dpartial{}{x} \left( \dpartial{f}{x} \right) $ -- dwukrotne różniczkowanie $f$ po $x$,
    \item $ \frac{\partial^2 f}{\partial y^2} = \dpartial{}{y} \left( \dpartial{f}{y} \right) $ -- dwukrotne różniczkowanie $f$ po $y$,
    \item $ \frac{\partial^2 f}{\partial y \partial x} = \dpartial{}{y} \left( \dpartial{f}{x} \right) $ -- różniczkowanie wpierw po $x$, potem po $y$,
    \item $ \frac{\partial^2 f}{\partial x \partial y} = \dpartial{}{x} \left( \dpartial{f}{y} \right) $ -- różniczkowanie wpierw po $y$, potem po $x$, \\
\end{itemize}

Inne oznaczenia to $ f_{xx}, f_{yy}, f_{xy}, f_{yx} $, gdzie indeks dolny oznacza zmienne, po których kolejno różniczkujemy.

W przypadku pochodnych mieszanych $ f_{xy}, f_{yx} $, trzeba ustalić kolejność różniczkowania.

Przyjmujemy naturalną kolejność, wtedy mamy $ f_{xy} = (f_x)_y $ \ oraz \ $ f_{yx} = (f_y)_x $,

co oznacza, że $ \frac{\partial^2 f}{\partial y \partial x} = f_{xy} $ \ i \ $ \frac{\partial^2 f}{\partial x \partial y} = f_{yx} $. \\

Dla funkcji $n$ zmiennych $ f = f(x_1, x_2, ..., x_n) $:
$$ \frac{\partial^2 f}{\partial x_j \partial x_i} = \dpartial{}{x_j} \left( \dpartial{f}{x_i} \right) = (f_{x_i})_{x_j} = f_{x_i x_j} $$ \\

\textbf{Przykład}

$$ f(x,y) = \frac{2^y}{x+1} $$

$$ \textrm{Tutaj} \quad  f_x = -\frac{2^y}{(x+1)^2}, \ f_y = \frac{2^y \ln 2}{x+1} \quad \textrm{oraz} $$

$$ f_{xx} = (f_x)_x = \left( - \frac{2^y}{(x+1)^2} \right)_x = -2^y \left( \frac{1}{(x+1)^2} \right)_x = 2^y \cdot \frac{2}{(x+1)^3} $$

$$ f_{yy} = (f_y)_y = \left( \frac{2^y \ln 2}{x+1} \right)_y = \frac{\ln2}{x+1} \cdot (2^y)_y = \frac{2^y \cdot (\ln2)^2}{x+1} $$

$$ f_{xy} = (f_x)_y = \left( - \frac{2^y}{(x+1)^2} \right)_y = \left( \frac{-1}{(x+1)^2} \right) \cdot (2^y)_y = - \frac{2^y \ln2}{(x+1)^2} $$

$$ f_{yx} = (f_y)_x = \left( \frac{2^y \ln2}{x+1} \right)_x = 2^y \ln2 \left( \frac{1}{x+1} \right)_x = 2^y \ln2 \left( \frac{-1}{(x+1)^2} \right) = - \frac{2^y \ln2}{(x+1)^2} $$ \\

Otrzymaliśmy $ f_{xy} = f_{yx} $.

Jest to szczególny przypadek znanego twierdzenia. \\

\textbf{Twierdzenie Schwarza o pochodnych mieszanych}

Gdy pochodne mieszane drugiego rzędu są funkcjami ciągłymi w danym punkcie to są w tym punkcie równe.

W praktyce dla funkcji regularnych warunek ciągłości drugiego rzędu występuje zawsze na całych dziedzinach
stąd prawie zawsze zobaczymy równość wzorów pochodnych mieszanych.


\subsection*{Zbieżność w $\mathbb{R}^k$ i granice funkcji wielu zmiennych}
\addcontentsline{toc}{subsection}{Zbieżność w $R^k$  i granice funkcji wielu zmiennych}

Rozpatrujemy ciąg wielu punktów $ P_n = (x_n, y_n) \in \mathbb{R}^2 $.

Równoważnie możemy myśleć o wektorach $ \vec{v} \in \mathbb{R}^2 $ biorąc wektory pozycyjne punktów $P_n$ czyli $\vec{v} = \vec{OP}_n$.

Niech teraz $ P_0 = (x_0, y_0) \in \mathbb{R}^2 $. Mówimy, że $ P_n \to P_0 $, gdy odległość między $P_n$ i $P_0$ zbiega $0$.

Formalnie
$$ \limn P_n = P_0 \ \Leftrightarrow \ \limn |\overrightarrow{P_0 P_n}| = 0
\ \Leftrightarrow \ \limn \sqrt{(x_n - x_0)^2 + (y_n - y_0)^2} = 0 $$

Podobnie, gdy
$$ P_n = (x_n, y_n, z_n) \in \mathbb{R}^3 \quad \textrm{i} \quad P_0 = (x_0, y_0, z_0) \in \mathbb{R}^3 $$

To definiujemy
$$ \limn P_n = P_0 \ \Leftrightarrow \ \limn |\overrightarrow{P_0 P_n}| = 0
\ \Leftrightarrow \ \limn \sqrt{(x_n - x_0)^2 + (y_n - y_0)^2 + (z_n - z_0)^2} = 0 $$

Analogicznie rozszerzamy tę definicję na przypadek $k$ -- wymiarowy. \\

Poniższe twierdzenie pokazuje, że zbieżność $ P_n \to P_0 $ może być zdefiniowana w równoważny sposób. \\

\textbf{Twierdzenie(zbieżność po współrzędnych)}

Gdy
$$ P_n = (x_n, y_n) \in \mathbb{R}^2 \quad \textrm{i} \quad P_0 = (x_0, y_0) \in \mathbb{R}^2 $$
to mamy równoważność
$$ \limn P_n = P_0 \ \Leftrightarrow \ \limn x_n = x_0 \ \land \ \limn y_n = y_0 $$ \\

Dowód 

Implikacja $ \Leftarrow $ wynika bezpośrednio z arytmetyki granic :

Jeżeli $ \limn x_n = x_0 \ \land \ \limn y_n = y_0 $ to
$$ \limn |\overrightarrow{P_0 P_n}| = \limn \sqrt{(x_n - x_0)^2 + (y_n - y_0)^2} = \sqrt{(x_n - x_0)^2 + (y_n - y_0)^2} = 0 $$

Zatem
$$ \limn P_n = P_0 $$ \\

Implikacja $ \Rightarrow $ wynika z kolei z twierdzenia o 3 funkcjach. 

Mamy bowiem
$$ 0 \leq |x_n - x_0| = \sqrt{(x_n - x_0)^2} \leq \sqrt{(x_n - x_0)^2 + (y_n - y_0)^2} = |\overrightarrow{P_0 P_n}| $$

Teraz, gdy $ \limn P_n = P_0 \quad \textrm{to} \quad \limn |\overrightarrow{P_0 P_n}| = 0 $
i z twierdzenia o 3 ciągach dostajemy $ \limn |x_n - x_0| = 0 $ a to daje
$ \limn (x_n - x_0) = 0 \ \Leftrightarrow \ \limn x_n = x_0 $

Analogicznie otrzymujemy $ \limn y_n = y_0 $ \\

Jak łatwo zauważyć, twierdzenie ma analogiczną postać w przypadku wyższych wymiarów. \\

\textbf{Definicja granicy funkcji dwóch zmiennych w punkcie}

$ \lim_{(x,y) \to (x_0,y_0)} f(x,y) = L \ \Leftrightarrow $ dla dowolnych ciągów punktów $ (x_n, y_n) \neq (x_0, y_0) $
i takich, że $ \limn (x_n, y_n) = (x_0, y_0) $ zachodzi równość $ \limn f(x_n, y_n) = L $. \\

Definicja jest analogiczna w przypadku funkcji większej ilości zmiennych.

Równoważny zapis tej granicy, zgodny ze znaczeniem twierdzenia o zbieżności po współrzędnych to
$$ \lim_{\substack{x \to x_0 \\ y \to y_0}} f(x,y) = L $$

Twierdzenie o granicach znane dla funkcji jednej zmiennej (arytmetyka granic, symbole nieoznaczone itd.) pozostają prawdziwe.

Główny problem -- nie da się bezpośrednio zastosować niektórych popularnych technik, np. reguły de l'Hospitala.

\subsection*{Popularne techniki liczenia granic funkcji wielu zmiennych}
\addcontentsline{toc}{subsection}{Popularne techniki liczenia granic funkcji wielu zmiennych}

\begin{enumerate}
    \item Twierdzenie o 3 funkcjach. Jeżeli dla wszystkich punktów $ P \in \mathbb{R}^k $ z pewnego sąsiedztwa punktu
    $ P_0 \in \mathbb{R}^k $ zachodzi nierówność 
    $$ d(P) \leq f(P) \leq g(P) \quad \textrm{i} \quad \lim_{P \to P_0} d(P) = \lim_{P \to P_0} g(P) = L \quad \textrm{to} \quad \lim_{P \to P_0} f(P) = L $$

    \item Sprowadzenie granicy do przypadku jednej zmiennej.
    
    Jeżeli istnieje nowa zmienna $ t = t(P) $ takie, że $ f(P) = g(t) $ oraz 
    $ \lim_{P \to P_0} t = t_0 $ \ i \linebreak \ $ \lim_{t \to t_0} g(t) = L $ \ to \ $ \lim_{P \to P_0} f(P) = L $ 

    \item \textbf{COŚ O BRAKU GRANICY XD}
    $ \lim_{P \to P_0} f(P) $ nie istnieje
\end{enumerate}

Przypadek 3 jest szczególnie częsty, gdy pojawia się symbol nieoznaczony.

W przypadku funkcji dwóch zmiennych najczęściej wybiera się ciągi punktów $P_n$ i $Q_n$ z dwóch różnych krzywych.

$P$ jest wtedy z wykresu jakiejś krzywej: $ y=g(x)$ \ lub \ $x=g(y)$.

$Q$ jest z wykresu innej krzywej: $y=h(x)$ \ lub \ $x=h(y)$.

Obie krzywe muszą spotykać się w punkcie granicznym $P_0$.

Wtedy granice \ $ \lim_{P \to P_0} f(P) $ \quad i \quad $ \lim_{Q \to P_0} f(Q) $ \ stają się granicami funkcji jednej zmiennej. \\

\textbf{Przykłady}

$$ \lim_{\substack{x \to 0 \\ y \to 0}} (x^2 + 4y^2) \cos \left( x - 5y + \frac{2}{x} \right) $$

Wiemy, że \ $ x^2 + 4y^2 \geq 0 $ \ oraz \ $ -1 \leq \cos \left( x - 5y \frac{2}{x} \right) \leq 1 $, \ a stąd
$$ -(x^2 + 4y^2) \leq (x^2 + 4y^2) \cos \left( x - 5y + \frac{2}{x} \right) \leq x^2 + 4y^2 $$

Ponieważ 
$$ \lim_{\substack{x \to 0 \\ y \to 0}} (x^2 + 4y^2) = 0 = \lim_{\substack{x \to 0 \\ y \to 0}} (-(x^2 + 4y^2)) $$
z twiedzenia o 3 ciągach otrzymujemy
$$ \lim_{\substack{x \to 0 \\ y \to 0}} (x^2 + 4y^2) \cos \left( x - 5y + \frac{2}{x} \right) = 0 $$ \\

$$ \lim_{\substack{x \to 1 \\ y \to 1 \\ z \to 0}} \frac{2x - y + z - 1 - \ln(2x - y + z)}{(2x - y + z - 1)^2} $$

Tutaj możemy podstawić $ t = 2x - y + z $. Wtedy $ \lim_{\substack{x \to 1 \\ y \to 1 \\ z \to 0}} t = 1 $

i mamy

$$ \lim_{\substack{x \to 1 \\ y \to 1 \\ z \to 0}} \frac{2x - y + z - 1 - \ln(2x - y + z)}{(2x - y + z - 1)^2}
= \lim_{t \to 1} \frac{t - 1 - \ln t}{(t-1)^2} \left[ \frac{0}{0} \right] \stackrel{[H]}{=} \frac{1}{2}$$ \\

\textbf{Kolejny przykład}

$$ \lim_{\substack{x \to 0 \\ y \to 0}} \frac{\tan (x^2 - y^2)}{x - y} $$

Tutaj znów jest granica typu $ \frac{0}{0} $. Po podstawieniu \ $ t = x^2 - y^2 $ \ mamy granicę podstawową
$ \lim_{t \to 0} \frac{\tan t}{t} = 1 $.

Stąd wniosek, że trzeba nasze wyrażenie rozbić na iloczyn: 
$ \lim_{\substack{x \to 0 \\ y \to 0}} \frac{\tan (x^2 - y^2)}{x^2 - y^2} \cdot \frac{x^2 - y^2}{x - y} $

Mamy wtedy
$$ \lim_{\substack{x \to 0 \\ y \to 0}} \frac{\tan (x^2 - y^2)}{x^2 - y^2} = \lim_{t \to 0} \frac{\tan t}{t} = 1 $$
Oraz
$$ \lim_{\substack{x \to 0 \\ y \to 0}} \frac{x^2 - y^2}{x - y} = \lim_{\substack{x \to 0 \\ y \to 0}} \frac{(x - y)(x + y)}{x - y}
= \lim_{\substack{x \to 0 \\ y \to 0}} (x + y) = 0 $$

Stąd 
$$ \lim_{\substack{x \to 0 \\ y \to 0}} \frac{\tan (x^2 - y^2)}{x - y} = 1 \cdot 0 = 0 $$ \\

\textbf{Kolejny przykład}

$$ \lim_{\substack{x \to 0 \\ y \to 0}} \frac{x}{y} $$

Tutaj wykażemy brak granicy

Rozpatrujemy 2 krzywe przechodzące przez $(0,0)$. Na przykład \ $ y = x $ \ oraz \ $ y = 2x $.

Biorąc \ $ y = x $ \ mamy
$$ \lim_{\substack{x \to 0 \\ y \to 0}} \frac{x}{y} = \lim_{x \to 0} \frac{x}{x} = 1 $$
Natomiast dla \ $ y = 2x $ mamy
$$ \lim_{\substack{x \to 0 \\ y \to 0}} \frac{x}{y} = \lim_{x \to 0} \frac{x}{2x} = \frac{1}{2} \neq 1 $$
Zatem granica nie istnieje \\

\subsection*{Ciągłość funkcji wielu zmiennych}
\addcontentsline{toc}{subsection}{Ciągłość funkcji wielu zmiennych}

Definicja jest analogiczna jak dla funkcji jednej zmiennej -- granica funkcji jest równa wartości.

Formalnie,

$f$ jest ciągła w punkcie $ P_0 \in D_f $, \ gdy \ $ \lim_{P \to P_0} f(P) = f(P_0) $,

$f$ jest ciągła na zbiorze $ A \subset D_f $ jeżeli jest ciągła we wszystkich punktach z $A$. \\

Twierdzenia dotyczące arytmetyki funkcji ciągłych są analogiczne jak w przypadku jednej zmiennej. \\

\textbf{Przykład}

Wyznaczyć zbiór punktów ciągłości funkcji

$$ f(x,y) = \left\{ \begin{aligned} 2x + y + 1, & \ x \geq 0 \\ 2y + x, & \ x < 0 \end{aligned} \right. $$

Tutaj rozpatrujemy dwa obszary -- dane warunkami $ x \geq 0 $ \ oraz \ $x < 0$.

Brzegiem obu obszarów jest prosta $ x = 0 $ (oś $Y$).

W punktach \ $ (x,y), \ x > 0 $, \ funkcja jest ciągła, bo jest równa elementarnej na zbiorze otwartym.

Podobnie dla $ x < 0 $..

Pozostaje zbadać ciągłość w punktach brzegowych czyli w $ P_0  = (0, y_0) $.

Ze względu na warunek definiujący zbiór, dla takich punktów zbieżności trzeba rozpatrzeć 2 możliwe typy punktów
$$ P = (x,y) \to P_0 \quad \textrm{dla} \quad x \geq 0 \ \textrm{oraz} \ x < 0 $$

Dla \ $ x \geq 0 $ \ mamy
$$ \lim_{\substack{x \to 0 \\ y \to y_0}} f(x,y) = \lim_{\substack{x \to 0 \\ y \to y_0}} (2x + y - 1) = y_0 - 1 $$

Dla \ $ x < 0 $ \ mamy
$$ \lim_{\substack{x \to 0 \\ y \to y_0}} f(x,y) = \lim_{\substack{x \to 0 \\ y \to y_0}} (2y + x) = 2y_0 $$

Ponadto $ f(0,y_0) = y_0 - 1 $

Stąd ciągłość w \ $ P_0 = (0,y_0) $ \ ma miejsce, gdy \ $ y_0 - 1 = 2y_0 $, \ a więc dla \ $ y_0 = -1$.

Wtedy dla dowolnego ciągu punktów \ $ P = (x,y) \to (0, -1) $ \ mamy
$$ \lim_{\substack{x \to 0 \\ y \to -1}} f(x,y) = f(0,-1) = -2 $$

Zatem zbiorem punktów ciągłości $f$ jest zbiór
$$ D = \{ (x,y) : x \neq 0 \} \cup \{ (0,-1) \} $$

Interpretacja geometryczna wykresu -- składa się z dwóch osobnych ukośnych półpłaszczyzn, które spotykają
się w punkcie $(0, -1) $.


\subsection*{Ekstrema funkcji dwóch zmiennych}
\addcontentsline{toc}{subsection}{Ekstrema funkcji dwóch zmiennych}

\textbf{Definicja}

$f$ ma w $ P = (x_0, y_0) \in D_f $ minimum lokalne gdy $ f(x_0, y_0) $ jest najmniejszą wartością $f$ na pewnym kole o środku w $P$.

$f$ ma w $ P = (x_0, y_0) \in D_f $ minimum lokalne gdy $ f(x_0, y_0) $ jest największą wartością $f$ na pewnym kole o środku w $P$. \\

Gdy ta wartość jest najmniejsza/największa na całej dziedzinie $f$ to mówimy o ekstremum (minimum, maksimum) \underline{globalnym}. \\

Na przykład funkcja $ f(x,y) = x^4 + y^6 $ ma w $(0,0)$ minimum i jest ono globalne, bo
$$ f(0,0) = 0 $$
a dla dowolnego $ (x,y) \neq (0,0) $ mamy $ f(x,y) = x^4 + y^6 > 0 $. \\

Wyznaczenie ekstremów z definicji rzadko kiedy się udaje, najczęściej szukamy ich z użyciem pochodnych cząstkowych.

Daje się to robić dla funkcji regularnych: na badanym zbiorze \textbf{pochodne pierwszego i drugiego rzędu istnieją i są ciągłe}. \\

\underline{Warunek konieczny istnienia ekstremum}: tzw. \underline{punkt stacjonarny} czyli

$ P = (x_0, y_0) $ taki, że 

$$ \left\{ \begin{aligned} f_x(x_0, y_0) = 0 \\ f_y(x_0, y_0) = 0  \end{aligned} \right. $$

\textbf{To jeszcze nie wystarcza!} To tylko mówi, że płaszczyzna styczna (gdy istnieje) jest równoległa do płaszczyzny $XY$.

\underline{Warunek dostateczny}. Liczymy w $P$ specjalny wyznacznik -- tzw. \underline{hesjan}.
$$ W = H(P) = H(x_0, y_0) = \begin{vmatrix} f_{xx}(x_0, y_0), & f_{xy}(x_0, y_0) \\ f_{yx}(x_0, y_0) & f_{yy}(x_0, y_0) \end{vmatrix} $$

Interpretacja: $H$ to "wykrywacz" ekstremum: mówi czy ekstremum jest czy nie. \\

\textbf{Twierdzenie}

Jeżeli w pewnym otoczeniu $ P = (x_0, y_0) $ pochodne pierwszego i drugiego rzędu funkcji $f$ istnieją i są ciągłe oraz
$ f_x(x_0, y_0) = f_y(x_0, y_0) = 0 $ to zachodzą poniższe własności.

\begin{itemize}
    \item Gdy $ H(x_0, y_0) > 0 $ to \textbf{jest ekstremum}. Wtedy gdy $ f_{xx}(x_0, y_0) > 0 $ to jest minimum,
    a gdy $ f_{xx}(x_0, y_0) < 0 $ to jest maksimum.
    \item Gdy $ H(x_0, y_0) < 0 $ to \textbf{nie ma ekstremum}.
    \item Gdy $ H(x_0, y_0) = 0 $ to \textbf{nic nie wiemy} -- metoda nie działa.
\end{itemize}

\textbf{Uwaga}

Można udowodnić, że gdy $ H(x_0, y_0) > 0 $ to $ f_{xx}(x_0, y_0) $ oraz $ f_{yy}(x_0, y_0) $ są jednocześnie obie dodatnie
lub obie ujemne.

Zatem przy sprawdzaniu typu ekstremum (minimum/maksimum) możemy patrzeć na dowolną z tych pochodnych. \\

\textbf{Przykłady}

\begin{enumerate}
    \item $ f(x,y) = 2x^2 + 3y^2 $
    
    Mamy $ D_f = \mathbb{R}^2 $ oraz
    $$ f_x = 4x, \quad f_y = 6y $$
    Stąd
    $$ f_x = f_y = 0 \ \Leftrightarrow \ x = y = 0 \quad \textrm{czyli punkt standardowy to} \quad P = (0,0) $$
    Teraz
    $$ f_{xx} = 4, \ f_{yy} = 6, \ f_{xy} = f_{yx} = 0 $$
    To daje
    $$ W = H(0,0) = \begin{vmatrix} 4 & 0 \\ 0 & 6 \end{vmatrix} = 24 > 0 \quad \textrm{-- jest ekstremum} $$
    $ f_{xx}(0,0) = 4 > 0 $ więc w $(0,0)$ jest minimum $ f(0,0) = 0 $. 
    \bigskip
    \item $ f(x,y) = (x^2 - y^2)e^x $
    
    Mamy $ D_f = \mathbb{R}^2 $ oraz
    $$ f_x = 2xe^x + (x^2 - y^2)e^x = e^x(x^2 - y^2 + 2x) $$
    $$ f_y = -2ye^x $$

    Stąd
    $$ f_x = f_y = 0 \ \Leftrightarrow \ \begin{cases} x^2 - y^2 + 2x = 0 \\ y = 0 \end{cases} \Leftrightarrow \
    \begin{cases} y = 0 \\ x^2 + 2x = 0 \end{cases} \Leftrightarrow \ \begin{cases} y = 0 \\ x = 0 \ \lor \ x=-2 \end{cases} $$

    Czyli punkty stacjonarne to $ P_1 = (0,0), \ P_2 = (-2, 0) $.

    Teraz
    $$ f_{xx} = e^x(2x + 2) + e^x(x^2 - y^2 + 2x) $$
    $$ f_{yy} = -2e^x $$
    $$ f_{xy} = f_{yx} = -2ye^x $$

    Dla $ P_1 = (0,0) $ mamy
    $$ W = H(0,0) = \begin{vmatrix} 2 & 0 \\ 0 & -2 \end{vmatrix} = -4 < 0 \quad \textrm{Brak ekstremum w } P_1 $$

    Dla $ P_2 = (-2, 0) $ mamy
    $$ W = H(0,0) = \begin{vmatrix} -2e^2 & 0 \\ 0 & -2e^2 \end{vmatrix} = -4e^{-2} \cdot e^{-2} > 0 \quad \textrm{Jest ekstremum} $$

    $ f_{xx}(-2, 0) = -2e^{-2} < 0 $ \ czyli mamy maksimum o wartości \ $ f(-2, 0) = 4e^{-2} $.
\end{enumerate}


\subsection*{Ekstrema warunkowe}
\addcontentsline{toc}{subsection}{Ekstrema warunkowe}

\textbf{Definicja}

Funkcją warunkową nazwiemy \ $ f = f(x,y) $ \ gdzie dziedziną jest zbiór, który jest krzywą na płaszczyźnie $XY$ czyli ma postać zależności
między $x$ i $y$: \ $F(x,y) = 0$.

\underline{Interpretacja geometryczna:} taka funkcja $f$ to krzywa w przestrzeni $ \mathbb{R}^3 $ położona "pionowo pod/nad" krzywą na płaszczyźnie
daną równaniem \ $F(x,y) = 0$.

Zatem jest to zbiór punktów \ $ (x, y, f(x,y)) $, \ gdzie \ $ F(x,y) = 0 $.

Rzutem tej krzywej na płaszczyznę $XY$ jest krzywa płaska o równaniu $F(x,y) = 0$

\begin{center}
\includegraphics[scale=0.4]{img/rzut_plaszczyznaXY.png}
\end{center}

\textbf{Przykład}

\[ f(x,y) = -xy, \ F(x,y) = 2x + y = 0 \]

\begin{center}To daje \ $ y = -2x $ \ czyli\end{center} 
\[ f(x,y) = f(x, -2x) = 2x^2, \ y=-2x, \ x\in \mathbb{R} \]

\begin{center}Czyli zbiór punktów\end{center}
\[ (x, -2x, 2x^2), \ x \in \mathbb{R} \]

Jest to paraboloida ustawiona "pionowo" ale nad prostą \ $ y= -2x $ \ (w płaszczyźnie równoległej do osi $Z$ i zawierającej tą prostą).

\begin{center}
\includegraphics[scale=0.5]{img/paraboloida_przyklad.png}
\end{center}

Ekstrema warunkowe to ekstrema takich funkcji. Liczymy je metodami poznanymi z Analizy Matematycznej 1 (mamy funkcję 1 zmiennej).

W naszym przykładzie mamy do analizy funkcję \ $ f(x) = x^2, \ x \in \mathbb{R} $

Nie trzeba pochodnych, ekstremum to punkt dla \ $x = 0$ -- jest to minimum.

To daje \ $ y = -2 \cdot 0 = 0 $ \ oraz \ $ z = f(0,0) = 0 $ \ więc punkt $ (0,0,0) $.

\subsection*{Wartości największe i najmniejsze funkcji na zadanych zbiorach}

Mamy funkcję \ $ f(x,y), \ D_f = D $

Interesuje nas wartość największa i wartość najmniejsza $f$ na $D$. Te wartości mogą istnieć lub nie. To zależy od funkcji i zbioru.
\bigskip

\textbf{Twierdzenie (Wersja tw. Weiertrassa (AM1) dla funkcji dwóch zmiennych)}

Gdy $D$ jest domknięty (czyli cały brzeg $D$ jest zawarty w $D$) oraz ograniczony (czyli zawiera się w pewnym kole) i $f$ jest ciągła
na $D$ to wartość największa i wartość najmniejsza $f$ na $D$ jest osiągana.
\bigskip

Gdzie te wartości mogą być osiągane dla funkcji różniczkowalnych?
\begin{itemize}
    \item W punktach stacjonarnych $f$: $ f_x = f_y = 0 $.
    \item Na brzegu $D$: prowadzi to do funkcji warunkowych i ich wartości największych/najmniejszych -- jak dla funkcji jednej zmiennej w AM1
\end{itemize}
\bigskip

Dla punktów z obu przypadków liczymy wartości $f$ i z tych wartości wybieramy najwiekszą i najmniejszą. To daje odpowiedź.

\underline{Uwaga}: Dla punktów stacjonarnych \underline{nie trzeba sprawdzać czy jest to ekstremum}.

Nie potrzeba hesjanu itd. Wystarczy policzyć wartość.
\bigskip

\textbf{Przykład}
\[ f(x,y) = xy^2, \ x^2 + y^3 \leq 3 \]

Punkty stacjonarne

$ f_x = y^2 = 0 $

$ f_y = 2xy = 0 $

Wychodzą punkty $ (x, 0) $ oraz $ f(x,0) = \textcolor{magenta}{0} $

Brzeg: \ $ x^2 + y^2 = 3 $. Wystarczy wyliczyć \ $ y^2 = 3 - x^2 $ \ i to daje
\[f(x,y) = f(x) = x(3-x^2) = 3x - x^3, \ x\in \left[-\sqrt3, \sqrt3 \right] \]

Zadanie staje się zadaniem z AM1: znaleźć wartość największą/najmniejszą tej funkcji.

Zatem

\[ f(\pm \sqrt3) = \textcolor{magenta}{0} \]
\[ f' = 3 - 3x^2 = 0 \ \Leftrightarrow \ x = \pm 1 \in \left[ -\sqrt3, \sqrt3 \right] \]
\[ f(-1) = \textcolor{magenta}{-2}, \ f(1) = \textcolor{magenta}{2} \]

Stąd wartość największa to 2, jest osiągana w punktach \ $(1, \sqrt2) $ \ oraz \ $ (1, -\sqrt2) $.

Najmniejsza wartość to -2, jest osiągana w punktach \ $(-1, \sqrt2) $ \ oraz \ $(-1, -\sqrt2) $.


\subsection*{Zadania optymalizacyjne}
\addcontentsline{toc}{subsection}{Zadania optymalizacyjne}

Schemat taki jak w AM1.
\begin{enumerate}
    \item Ułożyć funkcję opisującą daną wielkość.
    \item Znaleźć dziedzinę tej funkcji pasującą do zadania (niekoniecznie dziedzinę naturalną).
    \item Znaleźć wartość największą lub najmniejszą tej funkcji na zadanej dziedzinie.
\end{enumerate}
\bigskip

\textbf{Przykład}

Spośród wszystkich trójkątów o obwodzie równym $3$ jednostki znaleźć ten trójkąt, który ma największe pole.
\medskip

1. Wzór funkcji.

Jeśli boki tego trójkąta mają długości $a,b,c > 0$ to pole jest dane wzorem
\[ S = \sqrt{p(p-a)(p-b)(p-c)} \textrm{ \ gdzie \ } p = \frac{a+b+c}{2} \quad \textrm{(wzór Herona)} \]
\medskip

2. Dziedzina $f$: $ a,b > 0, \ c = 3 - a - b > 0 $ \ oraz z warunku trójkąta

\[ \begin{array}{ccc} 
    a+b>c & \Leftrightarrow & b > 1,5 - a \\
    a+c>b & \Leftrightarrow & 0 < b < 1,5 \\
    b+c>a & \Leftrightarrow & 0 < a < 1,5
\end{array} \]

To daje trójkąt o wierzchołkach w punktach \ $(1.5, \ 0), (0, \ 1.5)$ oraz $(1.5, \ 1.5)$ \ ale bez brzegu.
Aby mieć gwarancję istnienia wartości największej (twierdznie Weiertrassa) dołączamy brzeg do trójkąta i mamy $D_f$:

\[ 0 \leq a \leq 1,5 \]
\[ 1,5 - a \leq b \leq 1,5 \]

\begin{center}
    \includegraphics[scale=0.5]{img/trojkat.png}
\end{center}
\bigskip

3.Wartość największa $f$ \bigskip

a) Na brzegu

Brzeg składa się z trzech boków o równaniach
\[ \begin{array}{cc}
    a = 1,5: & f \equiv \textcolor{magenta}{0} \\
    b = 1,5: & f \equiv \textcolor{magenta}{0} \\
    b = 1,5 - a: & f \equiv \textcolor{magenta}{0} \\
\end{array}
\]

To na pewno nie jest wartość największa \bigskip

b) W punktach stacjonarnych we wnętrzu

\[ f(a,b) = \sqrt{1,5(1,5-a)(1,5-b)(a+b-1,5)} \quad \textrm{więc} \]
\[ f_a = \frac{1}{2\sqrt{do poprawy}} \cdot 1,5 \cdot (1,5 - b) \cdot (-(a+b - 1,5) + 1(1,5 - a)) = 0 \]

To daje układ

\[ \begin{cases} (1,5 - b) \cdot (3 - 2a - b) = 0 \\ (1,5 - a) \cdot (3 - 2b - a) = 0 \end{cases} \]

Zatem 

\[ \begin{cases} b = 1,5 \ \textrm{brzeg -- odrzucamy} \ \lor \ 3 - 2a - b = 0 \\
    a = 1,5 \ \textrm{brzeg -- odrzucamy} \ \lor \ 3 - 2b - a = 0
\end{cases} \]

Dla punktów we wnętrzu trójkąta jest więc

\[ 
\begin{cases}
    3 - 2a - b = 0 \\
    3 - 2b - a = 0
\end{cases}    
\]

To daje \ $ a = b = 1$ oraz \ $ f(1,1) = \textcolor{magenta}{\frac{\sqrt3}{4}} $. To jest wartość największa.

Ponadto wtedy $c = 1$. Jest to więc trójkąt równoboczny.