\section{Całki niewłaściwe II rodzaju}

Ustalamy liczby $ a,b \in \mathbb{R}, \ a < b $. Niech $f$ będzie funkcją całkowalną na każdym przedziale postaci $[a, T] $,
gdzie $ a < T < b $. Definiujemy \underline{całkę niewłaściwą drugiego rodzaju} z $f$ na przedziale $[a, b)$ jako

$$ \int\limits_{a}^{b} f(x) \,dx = \lim_{T \to b^+} \int\limits_{a}^{T} f(x) \,dx, \quad \textrm{gdy granica po prawej stronie istnieje.}$$

Analogicznie, gdy $f$ jest całkowalna na każdym przedziale postaci $[T,b]$, gdzie $ a < T < b $. to definiujemy
\underline{całkę niewłaściwą pierwszego rodzaju} z $f$ na przedziale $(a, b]$ jako

$$ \int\limits_{a}^{b} f(x) \,dx = \lim_{T \to a^+} \int\limits_{T}^{b} f(x) \,dx, \quad \textrm{gdy granica po prawej stronie istnieje.}$$\\

Terminologia dotycząca takich całek jest taka, jak dla całek niewłaściwych 1 rodzaju. Są 3 przypadki : 

\begin{enumerate}
    \item Granica z prawej strony jest liczbą. Wtedy całka jest \underline{zbieżna} (do tej granicy).
    \item Granica z prawej strony jest równa $\infty$ lub $-\infty$. Wtedy całka jest \underline{rozbieżna} do $\infty$ lub $-\infty$.
    \item Granica z prawej strony nie istnieje. Wtedy mówimy, że całka jest \underline{rozbieżna}. \\
\end{enumerate}

Interpretacja geometryczna. \\

Podobnie jak dla zwykłej całki oznaczonej, jeżeli $f \geq 0$ na $(a,b]$ lub $[a,b)$ to całka niewłaściwa 2 rodzaju
$ \int\limits_{a}^{b} f(x) \,dx $ daje pole obszaru ograniczonego osią X, wykresem $f$ oraz prostymi $x=a$ oraz $x=b$.

Najczęściej definiujemy tego typu całkę w przypadku gdy $f$ ma asymptotę pionową $x=a$ lub $x=b$. Wtedy ten obszar
nie jest ograniczony z góry bądź z dołu. \\

Na przykład

$$ \int\limits_{0}^{1} \frac{1}{\sqrt{x}} \,dx = \lim_{T \to 0^+} \int\limits_{T}^{1} \frac{1}{\sqrt{x}} \,dx
\lim_{T \to 0^+} [2\sqrt{x}]_T^1 = \lim_{T \to 0^+} (2 - 2\sqrt{T}) = 2 $$

Całka jest zbieżna do 2. \\

\underline{Wersja całki obustronnej} \\

Ustalamy liczby $ a,b,c \in \mathbb{R}, \ a < c < b $. Niech $f$ będzie funkcją całkowalną na każdym przedziale postaci
$[a,T]$, $T < c$, oraz $ [T, b] $, $ T > c $. Definiujemy całkę niewłaściwą 2 rodzaju z $f$ na zbiorze $[a, c)\cup(c, b] $
jako sumę dwóch całek niewłaściwych. tzn.

$$ \int\limits_{a}^{b} f(x) \,dx = \int\limits_{a}^{c} f(x) \,dx + \int\limits_{c}^{b} f(x) \,dx $$

przy czym gdy przynajmniej jedna z całek z prawej strony nie istnieje lub zachodzi przypadek $ \infty - \infty $ to
$ \int\limits_{a}^{b} f(x) \,dx $ jest rozbieżna, a w pozostałych przypadkach całka ma wartość wynikającą z arytmetyki granic. \\

Najczęściej takie całki pojawiają się, gdy $f$ ma asymptotę w $x = c$. \\

\textbf{Twierdzenie}

Istnieją podstawienia, które każdą całkę niewłaściwą 2 rodzaju sprowadzają do przypadku całki niewłaściwej 1 rodzaju.

W szczególności

\begin{itemize}
    \item dla całki $(a,b]$ możemy wziąć $ t = \frac{1}{x - a} $ co daje $ x = a + \frac{1}{t} $ oraz
    $$ \int\limits_{a}^{b} f(x) \,dx = \int\limits_{C}^{\infty} \frac{1}{t^2} f \left(a + \frac{1}{t} \right) dt \quad
    \textrm{, gdzie} \quad C = \frac{1}{b - a} $$

    \item dla całki na $[a,b)$ możemy wziąć $t = \frac{1}{b - x}$ co daje $ t = b - \frac{1}{t} $ oraz
    $$ \int\limits_{a}^{b} f(x) \,dx = \int\limits_{C}^{\infty} \frac{1}{t^2} f \left(b - \frac{1}{t} \right) dt \quad
    \textrm{, gdzie} \quad C = \frac{1}{b - a} $$ \\
\end{itemize}

Na przykład dla $ p > 0 $ biorąc $ t = \frac{1}{x} $ mamy 

$$ \int\limits_{0}^{b} \frac{1}{x^p} \,dx = \int\limits_{\frac{1}{b}}^{\infty} \frac{1}{t^2} \cdot \frac{1}{ \left( \frac{1}{t} \right)^p } dt
= \int\limits_{\frac{1}{b}}^{\infty} \frac{1}{t^{2 - p}} \,dt $$

Podstawienie to oznacza też, że mamy analogiczne kryteria zbieżności dla całek 2 rodzaju - porównawcze i ilorazowe, przy
czym dla kryterium ilorazowego liczymy granicę ilorazu funkcji w odpowiednim końcu zadanego przedziału. \\

Na koniec, wartość główna całki $ \int\limits_{a}^{b} f(x) \,dx $ na $[a,c)\cup(c,b]$ to wielkość

$$ \textrm{P.V.} \int\limits_{a}^{b} f(x) \,dx = \lim_{T \to 0^+} 
\left( \int\limits_{a}^{c - T} f(x) \,dx + \int\limits_{c + T}^{b} f(x) \,dx \right) $$

o ile powyższa granica istnieje.

Oznacza to, że odpowiednie końce przedziałów całkowania są w jednakowej odległości od c i zbiegają do c.

\subsection*{Zbieżność bezwzględna całek niewłaściwych}
\addcontentsline{toc}{subsection}{Zbieżność bezwzględna całek niewłaściwych}

Definicja. Całka $ \int\limits_{a}^{\infty} f(x) \,dx $ jest \underline{zbieżna bezwzględnie}, gdy zbieżna jest całka
$ \int\limits_{a}^{\infty} |f(x)| \,dx $.

Analogiczne definicje mamy dla pozostałych całek 1 rodzaju oraz dla całek 2 rodzaju. \\

Uwagi

\begin{itemize}
    \item Gdy $f$ jest nieujemna to mamy $ \int\limits_{a}^{\infty} f(x) \,dx = \int\limits_{a}^{\infty} |f(x)| \,dx $
    i definicja nie wnosi nic nowego. Sytuacja się zmienia, gdy są przedziały na którym $f$ ma różne znaki.
    
    \item Nierówność $ \left| \int\limits_{a}^{T} f(x) \,dx \right| \leq \int\limits_{a}^{T} |f(x)| \,dx $ daje
    $ \left| \int\limits_{a}^{\infty} f(x) \,dx \right| \leq \int\limits_{a}^{\infty} |f(x)| dx $ ale gdy są przedziały
    na którym $f$ ma różne znaki to równość nie zachodzi.
    Zatem, ogólnie, $ \left| \int\limits_{a}^{\infty} f(x) \,dx \right| $ i $ \int\limits_{a}^{\infty} |f(x)| \,dx $
    \textbf{to nie to samo}. \\
\end{itemize}

\textbf{Twierdzenie}

Jeżeli całka niewłaściwa jest bezwzględnie zbieżna to jest zbieżna (w zwykłym sensie).

Transpozycja tego twierdzenia daje warunek równoważny : 

Jeżeli całka $ \int\limits_{a}^{\infty} f(x) \,dx $ nie jest zbieżna to również nie jest zbieżna bezwzględnie,

co oznacza $ \int\limits_{a}^{\infty} |f(x)| \,dx = \infty $.

Analogicznie dla pozostałych typów całek niewłaściwych. \\

Twierdzenie odwrotne nie jest prawdziwe. Są całki zbieżne ale nie bezwzględnie, np. $ \int\limits_{1}^{\infty} \frac{\sin x}{x} \,dx $.
Takie całki to tzw. całki \underline{zbieżne warunkowo}.

Są więc 3 możliwe sytuacje - 3 rozłączne podzbiory całek niewłaściwych:

\begin{center}
\includegraphics[scale=0.6]{rozbiezneirozbiezne.png}
\end{center}

Przykład 

Całka $ \int\limits_{1}^{\infty} \frac{\sin x}{\sqrt[3]{x^4}} \,dx $ jest zbieżna bezwzględnie, bo biorąc
$ \int\limits_{1}^{\infty} \left| \frac{\sin x}{\sqrt[3]{x^4}} \right| \,dx $ i używając kryterium porównawczego mamy

$$ 0 \leq \left| \frac{\sin x}{\sqrt[3]{x^4}} \right| = \frac{|\sin x|}{x^{\frac{4}{3}}} \leq \frac{1}{x^{\frac{4}{3}}} $$

a całka $ \int\limits_{1}^{\infty} \frac{1}{x^{\frac{4}{3}}} \,dx $ jest zbieżna bo $ \frac{4}{3} > 1 $.
Zatem $ \int\limits_{1}^{\infty} \left| \frac{\sin x}{\sqrt[3]{x^4}} \right| \,dx $ jest zbieżna, a stąd
$ \int\limits_{1}^{\infty} \frac{\sin x}{\sqrt[3]{x^4}} \,dx $ też jest zbieżna.