\section{Transformata Laplace'a}

\begin{tw}{Definicja}
    Dana jest funkcja $f$ określona i całkowalna na półprostej $ [0, \infty) $.

    \underline{Transformata Laplace'a} funkcji $f$ to funkcja
    \[ F(s) = \laplace \{ f(t) \} = \int\limits_{0}^{\infty} e^{-st} f(t) \, dt \]

    a jej dziedziną jest zbiór tych $s$ dla których powyższa całka jest zbieżna.
\end{tw}

\begin{przyklad}
    Dla funkcji stałej $f = 1$ mamy dla $s > 0$
    \[ F(s) = \int\limits_{0}^{\infty} e^{-st} \cdot 1 \, dt = \lim_{T \to \infty} \int\limits_{0}^{T} e^{-st} \, dt =
    \lim_{T \to \infty} \left[ \frac{e^{-st}}{-s} \right]_{0}^{T} = \lim_{T \to \infty} \left( \frac{e^{-sT}}{-s} + \frac{1}{s} \right) = \frac{1}{s} \]

    Natomiast dla $s \leq 0$ ta całka jest rozbieżna. Stąd $ D_F = (0, \infty) $.

    Z kolei funkcja $f(t) = e^{t^2}$ nie posiada transformaty Laplace'a, gdyż "rośnie zbyt szybko" - całka

    \[ \int\limits_{0}^{\infty} e^{-st} \cdot e^{t^2} \, dt = \int\limits_{0}^{\infty} e^{t^2 - st} \, dt \]
    jest rozbieżna do $\infty$ dla wszystkich $s$.
\end{przyklad}

Popularnymi funkcjami dla których transformata Laplace'a istnieje są funkcje spełniające warunki poniższego twierdzenia.

\begin{tw}{Twierdzenie}
    Zakładamy, że funkcja $f$ określona na półprostej $ [0, \infty) $ spełnia poniższe warunki.
    \begin{enumerate}
        \item Na każdym przedziale postaci $ [0,T] $ jest ciągła lub ma skończoną ilośc punktów nieciągłości i nieciągłości te są pierwszego rodzaju
        \item $ \exists C, \ \alpha > 0 \ \forall t \geq 0 \quad |f(t)| \leq Ce^{\alpha i} $      
    \end{enumerate}
    Wtedy dla $s > \alpha$ istnieje transformata Laplace'a funkcji $f$.

    Funkcja spełniająca powyższe dwa warunki jest nazywana \underline{oryginałem}, a jej transformata -- \underline{obrazem} $f$.
\end{tw}

\pagebreak

Popularne funkcje i ich transformaty Laplace'a

\begin{table}[!htbp]
    \centering
    \begin{tabularx}{\linewidth}{|C|C|}
        \hline
        $f$ & $\laplace$ \\[15pt] \hline
        $ C, \ C \in \mathbb{R} $ & $ \frac{C}{s} $ \\[15pt] \hline
        $ t^n, \ n \in \mathbb{N} $ & $ \frac{n!}{s^{n+1}} $ \\[15pt] \hline
        $ e^{\alpha t} $ & $ \frac{1}{s - \alpha} $ \\[15pt] \hline
        $ \sin(\beta t) $ & $ \frac{\beta}{s^2 + \beta^2} $ \\[15pt] \hline
        $ \cos(\beta t) $ & $ \frac{s}{s^2 + \beta^2} $ \\[15pt] \hline
        $ \sinh(\beta t) $ & $ \frac{\beta}{s^2 - \beta^2} $ \\[15pt] \hline
        $ \cosh(\beta t) $ & $ \frac{s}{s^2 - \beta^2} $ \\[15pt] \hline
        $ t^n e^{\alpha t}, \ n \in \mathbb{N} $ & $ \frac{n!}{(s-\alpha)^{n+1}} $ \\[15pt] \hline
        $ e^{\alpha t} \sin(\beta t) $ & $ \frac{\beta}{(s - \alpha)^2 + \beta^2} $ \\[15pt] \hline
        $ e^{\alpha t} \cos(\beta t) $ & $ \frac{s - \alpha}{(s - \alpha)^2 + \beta^2} $ \\[15pt] \hline
    \end{tabularx}
\end{table} \medskip

Dziedziny tych transformat to odpowiednie półproste

\pagebreak

\subsection{Podstawowe własności transformaty Laplace'a}

\begin{tw}{Własności}
    Zakładamy, że dla funkcji $f$ i $g$ istnieją ich obrazy $Lf = L\{ f \} = \laplace \{ f(t) \} $ 

    oraz $ Lg = L\{ g \} = \laplace \{ g(t) \} $.

    Wtedy zachodzą własności opisane poniżej

    \begin{enumerate}
        \item (jednoznaczność) Gdy $f$ i $g$ są ciągłe i $Lf = Lg$ to $f = g$.
        \item (liniowość) $ L\{ f \pm g \} = Lf \pm Lg $ oraz $ L \{ c \cdot f \} = c \cdot Lf, \ c\in \mathbb{R} $.
        \item (skalowanie) Jeżeli $ g(t) = f(at), \ a \neq 0, $ to $ Lg(s) = \frac{1}{a} Lf \left( \frac{s}{a} \right) $.
        \item (przesunięcie obrazu) Jeżeli $ g(t) = e^{at} $ to $ L \{ f \cdot g \}(s) = Lf(s - a)$.
        \item (przesunięcie oryginału) Jeżeli dla $a > 0$
        
        \[ g(t) = \begin{cases}
            f(t - a), t \geq a \\ 0, \ t < a
        \end{cases} \]

        to $Lg(s) = e^{-as} Lf(s) $

        \item (pochodna obrazu) Jeżeli $g(t) = t^n \cdot f(t), \ n \in \mathbb{N}^+$ to $Lg(s) = (-1)^n (Lf)^{(n)}(s)$.
        \item (transformata całki) Jeżeli $ g(t) = \int\limits_{0}^{t} f(u) \, du $ to $ Lg(s) = \frac{1}{s} Lf(s) $.
        \item (transformata pochodnej) Opisana jest poniżej, jako osobne twierdzenie
    \end{enumerate}

\end{tw}

\begin{tw}{Twierdzenie -- Transformata pochodnej}
    Niech $ n \in \mathbb{N}^+ $. Zakładamy, że na $ (0, \infty) $ istnieje $ f^{(n)} $ i jest ciągła oraz funkcje $ f, f', f'',..., f^{(n-1)} $ są 
    oryginałami mają skończone granice prawostronne w 0:
    \[ f(0^+) = \lim_{x \to 0^+} f(t) \]
    \[ f'(0^+) = \lim_{x \to 0^+} f'(t) \]
    \[ \vdots  \]
    \[ f^{(n - 1)} (0^+) = \lim_{x \to 0^+} f^{(n - 1)}(t) \]

    Wtedy zachodzą wzory: \medskip
    \[ Lf'(s) = sLf(s) - f(0^+) \]
    \[ Lf''(s) = s^2 Lf(s) - sf(0^+) - f'(0^+) \]
    \[ Lf^{(3)}(s) = s^3 Lf(s) - s^2 f(0^+) - sf'(0^+) - f''(0^+) \]
    \[ Lf^{(n)} = s^n Lf(s) - s^{n - 1} f(0^+) - s^{n - 2}f'(0^+) - ... - sf^{(n-2)}(0^+) - f^{(n-1)}(0) \]

    W szczególności, gdy $ f, f', f'', ..., f^{(n-1)} $ są ciągłe w 0 to ich granice prawostronne w 0 mogą być zastąpione przez wartości w 0:
    \[ Lf'(s) = sLf(s) - f(0) \]
    \[ Lf''(s) = s^2 Lf(s) - sf(0) - f'(0) \]
    \[ \vdots \]
    \[ Lf^{(n)}(s) = s^n Lf(s) - s^{n-1} f(0) - s^{n-2} f'(0) - ... - s f^{(n - 2)} (0) - f^{(n - 1)} f(0) \]
\end{tw}

\begin{blad}{Uwaga}
    Wzór wynika z całkowania przez części i jest regularny. W odejmowanych iloczynach postaci "potęga s $\cdot$ pochodna w 0" zmniejszamy wkładnik $s$ o 1
    i jednocześnie zwiększamy rząd pochodnej o 1. Suma wykładnika dla $s$ i rzędu pochodnej w 0 jest stała i wynosi $n - 1$.
\end{blad}

\bigskip

Bez obliczania całek znaleźć transformaty poniższych funkcji
\begin{przyklad}
    1. $ f(t) = t^2 + 3t - 1 $
    
    Tutaj
    \[ Lf(s) = L\{ t^2 \} + L\{3t\} - L\{1\} = \frac{2!}{s^3} + 3L\{ t \} - \frac{1}{s} = \frac{2}{s^3} + 3 \cdot \frac{1!}{s^2} - \frac{1}{s} = \frac{2 + 3s - s^2}{s^3} \]
\end{przyklad}

\begin{przyklad}
    2. $ f(t) = \sin \left( t + \frac{\pi}{4} \right) $

    Wtedy
    \[ Lf(s) = L \left\{ \frac{\sqrt{2}}{2}(\sin t + \cos t) \right\} = \frac{\sqrt{2}}{2}(L \{ \sin t \} + L \{ \cos t \}) = \frac{\sqrt{2}}{2} \left( \frac{1}{s^2 + 1} + \frac{s}{s^2 + 1} \right) \]
    Analogicznie postępujemy z innymi funkcjami postaci $ \sin (t \pm \alpha) $ lub $ \cos (t \pm \alpha) $. 
\end{przyklad}

\begin{przyklad}
    3. $ f(t) = \cos^2 t $
    
    Mamy wzór $ \cos^2 t = \frac{1}{2} + \frac{1}{2} \cos(2t) $

    Stąd 
    \[ Lf(s) = L \frac{1}{2} + \frac{1}{2}L \{ \cos(2t) \} = \frac{1}{2} \cdot \frac{1}{s} + \frac{1}{2} \cdot \frac{s}{s^2 + 2^2} = \frac{1}{2} \left( \frac{1}{s} + \frac{s}{s^2 + 4} \right) \]
\end{przyklad}

\begin{przyklad}
    4. $ f(t) = t^2 \sin t $

    Ponieważ $ L \{ \sin t \} = \frac{1}{s^2 + 1} $ to zgodnie ze wzorem na pochodną obrazu dostajemy
    \[ L \{ t^2 \sin t \}(s) = (-1)^2 \cdot (L \{ \sin t \})'' (s) = \left( \frac{1}{s^2 + 1} \right)'' = \frac{2s^2 - 2}{(s^2 + 1)^3} \]
\end{przyklad}
\bigskip

Teraz na przykład odwrotną operację -- znaleźć $f$ mając daną jej transformatę.

\begin{przyklad}
    1. $ Lf(s) = \frac{s}{s^2 - 4} $

    Wzór jest postaci $ \frac{s}{s^2 - \text{liczba} > 0} $ więc jest związany z funkcją $\cosh$:
    \[ L \{ \cos (\beta t) \} = \frac{s}{s^2 - \beta^2} \]

    Mamy zatem bezpośrednio $ Lf(s) = \frac{s}{s^2 - 2^2} $, a stąd $ f(t) = \cosh(2t) $.
\end{przyklad}

\begin{przyklad}
    2. $ Lf(s) = \frac{1}{s^2 + 5} $

    Wzór jest postaci $ \frac{\text{stała}}{s^2 + \text{liczba} > 0} $ więc powinien być związany z sinusem, bo $ L\{ \sin (\beta t) \} = \frac{\beta}{s^2 + \beta^2} $

    Poprzez dopasowanie mamy
    \[ Lf(s) = \frac{1}{s^2 + 5} = \frac{1}{s^2 + (\sqrt{5})^2} = \frac{1}{\sqrt{5}} \cdot \frac{\sqrt{5}}{s^2 + (\sqrt{5})^2} \]
    Stąd \ $ \beta = \sqrt{5} $ \ oraz \ $ f(t) = \frac{1}{\sqrt{5}} \sin(t \sqrt{5}) $.
\end{przyklad}

\begin{przyklad}
    3. $ Lf(s) = \frac{1}{s^6} $

    Wzór jest postaci $ \frac{\text{stała}}{s^k}, \ k \in \mathbb{N} $. Zatem powinien być związany z wielomianem bo
    \[ L \{ t^n \} = \frac{n!}{s^{n + 1}} \]
    
    U nas \ $ n + 1 = 6 $ \ i przez dopasowanie mamy
    \[ Lf(s) = \frac{1}{s^6} = \frac{1}{5!} \cdot \frac{5!}{s^{5 + 1}} \]
    Stąd $ f(t) = \frac{1}{5!} t^5 = \frac{1}{120} t^5 $.
\end{przyklad}

\begin{przykladbig}
    4. $ Lf(s) = \frac{2s - 3}{s^2 + 6s + 13} $

    Mianownik nie ma pierwiastków więc jest to ułamek prosty drugiego rodzaju z czynnikiem jednokrotnym w mianowniku. Jest on zawsze trasnformatą funkcji typu
    \[ f(t) = e^{\alpha t} (A \cos (\beta t) + B \sin (\beta t)), \quad \text{gdzie} \quad \alpha, \beta, A, B \in \mathbb{R} \]

    Niektóre z tych stałych mogą być równe 0 i wtedy funkcja jest prostsza.

    Aby odtworzyć $f$ trzeba na początek zapisać mianownik w postaci kanonicznej
    \[ (s - \alpha)^2 + \beta^2 \]

    Następnie licznik zapisujemy w postaci
    \[ L = A(s - \alpha) + B \beta \]
    
    To pozwala rozbić ułamek na 2 części i wyznaczyć $f$:
    \[ \frac{L}{(s - \alpha)^2 + \beta^2} = \frac{A(s - \alpha) + B \beta}{(s - \alpha)^2 + \beta^2} = A \frac{s - \alpha}{(s - \alpha)^2 + \beta^2} + B \frac{\beta}{(s - \alpha)^2 + \beta^2} \]

    co daje wspomnianą funkcję $ f(t) = e^{\alpha t} (A \cos (\beta t) + B \sin (\beta t)) $.
    
    W naszym przykładzie mamy 
    \[ s^2 + 6s + 13 = (s+3)^2 + 4 = (s - (-3))^2 + 2^2 \]
    
    Stąd \ $ \alpha = -3, \ \beta = 2 $

    Dalej,
    \[ 2s - 3 = A(s + 3) + B \cdot 2 = As + 3A + 2B \quad \text{co daje} \quad A = 2, B = -\frac{9}{2} \]
    
    Zatem 
    \[ f(t) = e^{-3}\left( 2 \cos(2t) - \frac{9}{2} \sin(2t) \right) \]
\end{przykladbig}

\begin{blad}{Uwaga}
    Ułamki proste drugiego rodzaju z mianownikiem o krotności $k > 1$ wymagają $ k - 1 $ pochodnej z ułamków z czynnikiem jednokrotnym, co jest
    związane z pochodną transformaty i mnożeniem funkcji przez $ t^{k -1}$.
\end{blad}

Operacje opisane w powyższych przykładach najczęsciej pojawiają się przy jednym z najbardziej popularnych zastosowań transformaty Laplace'a -- rozwiązywania pewnych typów równań różniczkowych.