\documentclass[12pt]{article}
\usepackage[a4paper, total={6in, 8in}]{geometry}
\usepackage{amsfonts}
\usepackage{amsmath,amssymb,trimclip,adjustbox}
\usepackage{breqn}
\usepackage{hyperref}
\usepackage{tabularray}
\usepackage[dvipsnames]{xcolor} 
\usepackage{polski}
\usepackage[utf8]{inputenc}


\setlength{\parindent}{0pt}
\setlength{\textheight}{680pt}
\setlength{\oddsidemargin}{0pt}
\setlength{\textwidth}{480pt}

\everymath{\displaystyle}

\author{Skrypt wykładu Krzysztofa Michalika}
\title{Wykład 1 - Analiza matematyczna II}

\begin{document}
\maketitle

\section{Całki niewłaściwe I rodzaju}

Ustalamy liczbę $a \in \mathbb{R}$. Niech $f$ będzie funkcją całkowalną na każdym przedziale \linebreak w postaci $[a, T]$
gdzie $T > a$. Definiujemy \underline{całkę niewłaściwą pierwszego rodzaju} z $f$ na półprostej $[a, \infty]$ jako

$$ \int\limits_{a}^{\infty} f(x) \,dx = \lim_{T \to \infty} \int\limits_{a}^{T} f(x) \,dx \ \text{,  gdy granica po prawej stronie istnieje} $$

Analogicznie, gdy $f$ jest całkowalna na każdym przedziale postaci $[T, b]$, gdzie $T < b$. Definiujemy całkę niewłaściwą
pierwszego rodzaju z $f$ na półprostej [$-\infty$, b] jako

$$ \int\limits_{-\infty}^{b} f(x) \,dx = \lim_{T \to -\infty} \int\limits_{T}^{b} f(x) \,dx \ \text{,  gdy granica po prawej stronie istnieje} $$\\

Terminologia dotycząca takich całek jest taka, jak dla ciągów. Są 3 przypadki :

\begin{enumerate}
    \item Granica z prawej strony jest liczbą. Wtedy mówimy, że całka jest \underline{zbieżna}.
    \item Granica z prawej strony jest równa $\infty$ lub $-\infty$. Wtedy mówimy, że całka jest \underline{rozbieżna} (odpowiednio do $\infty$ lub $-\infty$).
    \item Granica z prawej strony nie istnieje. Wtedy mówimy, że całka jest \underline{rozbieżna}.
\end{enumerate}

Analogicznie dla $ \int\limits_{\infty}^{b} f(x)\,dx $

Przykłady :

$$ \int\limits_{0}^{\infty} \sin x \,dx = \lim_{T \to \infty} \int\limits_{0}^{T} \sin x \,dx = 
\lim_{T \to \infty} [-\cos x]_0^T = \lim_{T \to \infty} (-\cos T - (- \cos 0)) = \lim_{T \to \infty} (1 - \cos T) $$

Granica ta nie istnieje więc całka jest rozbieżna. \\

$$ \int\limits_{-\infty}^{0} 2^x \,dx = \lim_{T \to -\infty} \int\limits_{T}^{0} 2^x \,dx = 
\lim_{T \to -\infty} \left[ \frac{2^x}{\ln 2} \right]_T^0 = \lim_{T \to -\infty} 
\left( \frac{1}{\ln 2} - \frac{2^T}{\ln 2} \right) = \frac{1}{\ln 2} $$

Całka jest zbieżna do $ \frac{1}{\ln 2} $. \\

Pozostaje przypadek $ p = 1 $. Wtedy

$$ \int \frac{1}{x} \,dx = \ln |x| + C, \ \
\int\limits_{a}^{T} \frac{1}{x} \,dx = [\ln |x|]_a^T = \ln |T| - \ln |a|, \ \
\int\limits_{a}^{\infty} \frac{1}{x} \,dx = \lim_{T \to \infty} (\ln |T| - \ln |a|) = \infty $$

Udowodniliśmy zatem ważny wynik \\

\textbf{Twierdzenie} 

Gdy $ a > 0 $ to całka $ \int\limits_{a}^{\infty} \frac{1}{x^p} \,dx $
jest skończona dla $ p > 1 $ oraz nieskończona dla $ p \leq 1 $.

Podobnie można łatwo pokazać poniższy wynik \\

\textbf{Twierdzenie}

Gdy $ a \in \mathbb{R} $ i $ A > 0 $ to całka $ \int\limits_{a}^{\infty} A^x \,dx $
jest skończona dla $ 0 < A < 1 $ oraz nieskończona \linebreak dla $ A \geq 1 $ \\

Gdy $ \int f(x) \,dx = F(x) + C $ \ to

$$ \int\limits_{-\infty}^{\infty} f(x) \, dx = \lim_{T \to \infty} F(T) - \lim_{S \to \infty} F(S) $$

przy czym przynajmniej jedna z granic z prawej strony nie istnieje lub zachodzi przypadek 
$ \infty - \infty $ to $ \int\limits_{-\infty}^{\infty} f(x) \,dx $ jest rozbieżna, a w pozostałych
przypadkach całka ma wartość wynikającą z arytmetyki granic. \\

W przypadku kiedy całki nie da się obliczyć w sposób dokładny można to zrobić w sposób przybliżony, pod warunkiem
, że wiemy, że jest zbieżna.

Kryteria zbieżności to twierdzenia opisujące warunki dostateczne zbieżności lub rozbieżności danej klasy
całek. Najczęściej mają postać implikacji ale NIE równoważności. \\

Oznacza to zwykle własności postaci

\quad warunek zachodzi $ \Rightarrow $ całka jest zbieżna/rozbieżna

\quad warunek nie zachodzi $ \Rightarrow $ nic nie wiemy o zbieżności/rozbieżności całki

\subsection*{Popularne kryteria zbieżności całek z $\infty$}

0. Warunek konieczny zbieżności całki

Jeżeli całka $ \int\limits_{a}^{\infty} f(x) \,dx $ jest zbieżna to 
$ \lim\limits_{x \to \infty} f(x) $ jest równa 0 lub nie istnieje. \\

Transpozycja twierdzenia daje następujący wynik:

Jeżeli $ \lim\limits_{x \to \infty} f(x) $ istnieje i jest różna od 0 to całka 
$ \int\limits_{a}^{\infty} f(x) \, dx $ nie jest zbieżna, przy czym

\begin{itemize}
    \item gdy $ \lim\limits_{x \to \infty} f(x) > 0 $ to $ \int\limits_{a}^{\infty} f(x) \,dx = \infty $,
    \item gdy $ \lim\limits_{x \to \infty} f(x) < 0 $ to $ \int\limits_{a}^{\infty} f(x) \,dx = -\infty $,
\end{itemize}

\textbf{Uwaga. Warunek konieczny to tylko implikacja!}

Jeżeli $ \lim_{x \to \infty} f(x) $ jest równa 0 lub nie istnieje to jeszcze \textbf{NIC NIE WIEMY} o całce,

Na przykład całki $ \int\limits_{a}^{\infty} \frac{1}{x^p} \,dx, \ a > 0 $, mają
$ \lim_{x \to \infty} \frac{1}{x^p} = 0 $ dla wszystkich $ p > 0 $ ale niektóre z tych całek są zbieżne,
a niektóre rozbieżne \\

\subsection*{Ważna klasa całek - całki z funkcji nieujemnych}

$$ \int\limits_{a}^{\infty} f(x) \,dx, \ f \geq 0 $$

Wtedy $ \int\limits_{a}^{T} f(x) \, dx = F(T) - F(a) $ jest funkcją niemalejącą zmiennej T zatem całka
$ \int\limits_{a}^{\infty} f(x) \, dx = \lim_{T \to \infty} \int\limits_{a}^{T} f(x) \,dx $
zawsze istnieje. Może być to liczba lub $\infty$.

Zatem brak zbieżności takich całek oznacza rozbieżność do $\infty$. \\

Dla całek z funkcji nieujemnych mamy dwa kolejne kryteria zbieżności.

\begin{enumerate}
    \item Kryterium porównawcze
    \item Kryterium ilorazowe
\end{enumerate}

\subsection*{Twierdzenie(kryterium porównawcze)}

Dane są dwie całki $ \int\limits_{a}^{\infty} f(x) \,dx $ oraz
$ \int\limits_{a}^{\infty} g(x) \, dx $. Wtedy zachodzą następujące własności

\begin{enumerate}
    \item (Przypadek zbieżności). Gdy $ \forall x \geq x_0 \geq a \ \ 0 \leq f(x) \leq g(x) $ i $ \int\limits_{a}^{\infty} g(x) \,dx $
    jest zbieżna to $ \int\limits_{a}^{\infty} f(x) \,dx $ też jest zbieżna. Ponadto \
    $ 0 \leq \int\limits_{a}^{\infty} f(x) \,dx \leq \int\limits_{a}^{\infty} g(x) \,dx $
    
    \item (Przypadek rozbieżności) Gdy $ \forall x \geq x_0 \geq a \ \ 0 \leq g(x) \leq f(x) $ i $ \int\limits_{a}^{\infty} g(x) \,dx $
    jest rozbieżna (więc równa $\infty$) to $ \int\limits_{a}^{\infty} f(x) \,dx $ też jest rozbieżna (do $\infty$).
    
    \item \textbf{(Przypadek wątpliwy)} Gdy $ \forall x \geq x_0 \geq a \ \ 0 \leq f(x) \leq g(x) $ ale $ \int\limits_{a}^{\infty} g(x) \,dx $
    jest rozbieżna to \textbf{NIC NIE WIEMY} o zbieżności $ \int\limits_{a}^{\infty} f(x) \,dx $.
    
    \item \textbf{(Przypadek wątpliwy)} Gdy $ \forall x \geq x_0 \geq a \ \ 0 \leq g(x) \leq f(x) $ ale $ \int\limits_{a}^{\infty} g(x) \,dx $
    jest zbieżna to \textbf{NIC NIE WIEMY} o zbieżności $ \int\limits_{a}^{\infty} f(x) \,dx $.
\end{enumerate}

Uwagi:

\begin{itemize}
    \item $ \int\limits_{a}^{\infty} f(x) \,dx $ jest całką z zadania, $ \int\limits_{a}^{\infty} g(x) \,dx $ tworzymy sami.
    \item Porównujemy najczęściej z całkami $ \int\limits_{a}^{\infty} A^x \,dx $ lub 
    $ \int\limits_{a}^{\infty} \frac{1}{x^p} \,dx $. Wtedy $f$ często ma postać ułamków i możemy spróbować
    wziąć $g$ jako :

    \quad C - iloraz najwyższych potęg z licznika i mianownika $f$

    \item Trzeba uważać aby nierówność między $f$ i $g$ była prawdziwa i nie zapomnieć przypadku wątpliwego, bo wtedy
    \textbf{trzeba zaczynać od nowa}.

    \item Warto sprawdzić opisany wyżej iloraz najwyższych potęg i na tej podstawie przewidzieć czy chcemy
    udowodnić zbieżność czy rozbieżność. To pomaga skonstruować odpowiednią nierówność między $f$ i $g$.
\end{itemize}

\textbf{Popularny błąd} - odpowiedź na podstawie przypadku wątpliwego \\

Na przykład dla całki $ \int\limits_1^\infty \frac{1}{x + \sqrt{x}} \,dx $ : 

"Mamy \ $ 0 \leq \frac{1}{x + \sqrt{x}} \leq \frac{1}{x} $ \ i całka \ $ \int\limits_1^\infty \frac{1}{x} \,dx $
jest rozbieżna \textcolor{red}{zatem całka $ \int\limits_{1}^{\infty} \frac{1}{x + \sqrt{x}} \,dx $ jest rozbieżna.}"

GAME OVER... To jest przypadek nr 3 (wątpliwy) \\

Przykład

$$ \int\limits_4^\infty \frac{2x - 3}{x^3 - 1} \,dx $$

Przewidywanie zbieżności/rozbieżności

Najwyższe potęgi sugerują, że mając

$$ \frac{x}{x^3} = \frac{1}{x^2}, \quad \textrm{ a} \quad \int\limits_4^\infty \frac{1}{x^2} \,dx < \infty, \quad \textrm{bo} \quad 2 > 1 $$

Dowodzimy zbieżność. Trzeba mieć
$$ 0 \leq \frac{2x - 3}{x^3 - 1} \leq g(x) = C \cdot \frac{x}{x^3} $$

Jak w twierdzeniu o 3 ciągach

$$ 0 \leq \frac{2x}{x^3 - \frac{1}{2}x^3} = 4 \cdot \frac{x}{x^3} = 4 \cdot \frac{1}{x^2} $$

$$ \int\limits_4^\infty \frac{4}{x^2} \,dx = 4 \int\limits_4^\infty \frac{1}{x^2} \,dx < \infty
\quad \left(\frac{1}{2}x^3 > 1 \ \mathrm{dla} \ x \geq 4 \right) $$

\subsection*{Twierdzenie(kryterium ilorazowe)}

Dane są dwie całki $ \int\limits_a^\infty f(x) \,dx $ oraz $ \int\limits_{a}^{\infty} g(x) \,dx $. Ponadto

$$ \forall x \geq x_0 \geq a \quad f(x), g(x) > 0 $$

Jeżeli istnieje granica $ \lim_{x \to \infty} \frac{f(x)}{g(x)} $ i jest \underline{liczbą dodatnią} to wtedy obie całki
są zbieżne albo obie rozbieżne do $\infty$. \\

Uwagi
\begin{itemize}
    \item Funkcję $g$ tworzymy podobnie jak dla kryterium porównawczego
    \item Nie ma problemu z nierównościami :) ale za to trzeba umieć liczyć granice
    \item Granica nie może być ani 0 ani $\infty$: $ \lim_{x \to \infty} \frac{f(x)}{g(x)} \in (0, \infty) $
    \item Rozwiązanie \colorbox{yellow}{musi zawierać wniosek} "granica ilorazu jest liczbą dodatnią więc obie całki
    są zbieżne lub obie rozbieżne" - bez tego będzie niepełne.
    \item Kryterium zwykle jest wygodniejsze niż porównawcze ale są przykłady, które "idą" z porównawczego ale nie z
    ilorazowego, bo granica ilorazu nie istnieje

    Np. $ \int\limits_{1}^{\infty} \frac{2 + \sin x}{x} \,dx $
\end{itemize}

Przykłady \\

Poprzedni przykład raz jeszcze 

$$ \int\limits_4^\infty \frac{2x - 3}{x^3 - 1} \,dx $$

$$ f(x) = \frac{2x - 3}{x^3 - 1}, \quad x \geq 4 $$

$$ g(x) = \frac{x}{x^3} = \frac{1}{x^2} > 0 $$

$$ \lim_{x \to \infty} = \frac{f(x)}{g(x)} = \lim_{x \to \infty} \frac{x^2(2x - 3)}{x^3 - 1} = 2 $$

Obie całki zbieżne lub obie rozbieżne do $\infty$ \\

Przykłady o postaci funkcji złożonej $ \int\limits_{a}^{\infty} f(g(x)) \,dx $
gdzie $ \lim_{x \to \infty} g(x) = 0^+ $ oraz $ \lim_{x \to 0^+} f(x) = 0^+ $

Nową całką jest całka z funkcji wewnętrznej $ \int\limits_{a}^{\infty} g(x) \,dx $ \\

Liczymy granicę

$$ \lim_{x \to \infty} \frac{f(g(x))}{g(x)} = \lim_{t = g(x) \to 0^+} \frac{f(t)}{t} \left[ \frac{0}{0} \right] $$ \\

przy użyciu granic podstawowych lub reguły de l'Hospitala. \\

Na przykład $ \int\limits_{1}^{\infty} \left( 2^{\frac{1}{\sqrt{x}}} - 1 \right) \,dx$

$$ \int\limits_{1}^{\infty} \left( 2^{\frac{1}{\sqrt{x}}} - 1 \right) \,dx $$

$$ g(x) = \frac{1}{\sqrt{x}} > 0 $$

$$ f(x) = 2^x - 1 > 0 $$

$$ \lim_{x \to \infty} \frac{2^{\frac{1}{\sqrt{x}}} - 1}{\frac{1}{\sqrt{x}}} = \lim_{t \to 0^+} \frac{2^t - 1}{t}
\left[ \frac{0}{0} \right] = \ln 2 \in (0, \infty) $$

Obie całki zbieżne lub obie rozbieżne

$$ \int\limits_1^\infty \frac{1}{\sqrt{x}} \,dx = \int\limits_1^\infty \frac{1}{x^{\frac{1}{2}}} \,dx = \infty\ \quad
\textrm{bo} \quad \frac{1}{2} \leq 1 $$

\subsection*{Wartość główna całki niewłaściwej I rodzaju}

Całka $ \int\limits_{-\infty}^{\infty} x \,dx $ jest rozbiezna, gdyż jako suma całek prowadzi do symbolu $ \infty - \infty $:

$$ \int\limits_{-\infty}^{\infty} x \,dx = \int\limits_{-\infty}^{0} x \,dx + \int\limits_{0}^{\infty} x \,dx = -\infty + \infty $$

Intuicyjnie oczekwialibyśmy jednak, że jest ona równa 0 - funkcja podcałkowa jest nieparzysta czyli mamy "tyle funkcji
na + co na -", a więc wszystko powinno się wzajemnie zrównoważyć.

Aby taka całka miała sens trzeba nieco zmodyfikować jej definicję i wprowadzić pojęcie wartości głównej całki niewłaściwej (obustronnej).

Definicja. Wartość główna całki $ \int\limits_{-\infty}^{\infty} f(x) \,dx $ to wielkość

$$ \textrm{P.V.} \int\limits_{-\infty}^{\infty} f(x) \,dx = \lim_{T \to \infty} \int\limits_{-T}^{T} f(x) \,dx $$

o ile powyższa granica istnieje. \\

Oznacza to, że przybliżamy całkę po $\mathbb{R}$ całkami po przedziale symetrycznym względem 0.

P.V. jest skrótem od angielskiego "Principal Value". \\

Na przykład 

$$ \textrm{P.V.} \int\limits_{-\infty}^{\infty} x \,dx = \lim_{T \to \infty} \int\limits_{-T}^{T} x\,dx
= \lim_{T \to \infty} 0 = 0 $$

Zauważmy, że gdy $ \int f(x) \,dx = F(x) + C $ to

$$ \textrm{P.V.} \int\limits_{-\infty}^{\infty} f(x) \,dx = \lim_{T \to \infty} \int\limits_{-T}^{T} f(x) \,dx
= \lim_{T \to \infty} (F(T) - F(-T)) $$

Jeżeli teraz ma sens wyrażenie $ \lim_{T \to \infty} F(T ) - \lim_{T \to \infty} F(-T) $ to biorąc $ S = -T \to -\infty $ dostajemy

$$ \textrm{P.V.} \int\limits_{-\infty}^{\infty} f(x) \,dx = \lim_{T \to \infty} (F(T) - F(-T)) = 
\lim_{T \to \infty} F(T) - \lim_{T \to \infty} F(-T) = $$ $$ =  \lim_{T \to \infty} F(T) - \lim_{S \to -\infty} F(S)
= \int\limits_{-\infty}^{\infty} f(x) \,dx $$

Udowodniliśmy zatem poniższe twierdzenie. \\

Jeżeli całka $ \int\limits_{-\infty}^{\infty} f(x) \,dx $ istnieje w zwykłym sensie (jako suma odpowiednich całek jednostronnych
jest liczbą lub jedną z nieskończoności) to również jej wartość główna istnieje i jest równa tej całce.

Natomiast może się zdarzyć, że wartość główna całki istnieje ale sama całka jest rozbieżna (był przykład).

W szczególności gdy funkcja jest na $\mathbb{R}$ ciągła i nieparzysta to wartość główna całki z tej funkcji jest zawsze
0 niezależnie od zbieżności samej całki.

\section{Całki niewłaściwe II rodzaju}

Ustalamy liczby $ a,b \in \mathbb{R}, \ a < b $. Niech $f$ będzie funkcją całkowalną na każdym przedziale postaci $[a, T] $,
gdzie $ a < T < b $. Definiujemy \underline{całkę niewłaściwą drugiego rodzaju} z $f$ na przedziale $[a, b)$ jako

$$ \int\limits_{a}^{b} f(x) \,dx = \lim_{T \to b^+} \int\limits_{a}^{T} f(x) \,dx, \quad \textrm{gdy granica po prawej stronie istnieje.}$$

Analogicznie, gdy $f$ jest całkowalna na każdym przedziale postaci $[T,b]$, gdzie $ a < T < b $. to definiujemy
\underline{całkę niewłaściwą pierwszego rodzaju} z $f$ na przedziale $(a, b]$ jako

$$ \int\limits_{a}^{b} f(x) \,dx = \lim_{T \to a^+} \int\limits_{T}^{b} f(x) \,dx, \quad \textrm{gdy granica po prawej stronie istnieje.}$$\\

Terminologia dotycząca takich całek jest taka, jak dla całek niewłaściwych 1 rodzaju. Są 3 przypadki : 

\begin{enumerate}
    \item Granica z prawej strony jest liczbą. Wtedy całka jest \underline{zbieżna} (do tej granicy).
    \item Granica z prawej strony jest równa $\infty$ lub $-\infty$. Wtedy całka jest \underline{rozbieżna} do $\infty$ lub $-\infty$.
    \item Granica z prawej strony nie istnieje. Wtedy mówimy, że całka jest \underline{rozbieżna}. \\
\end{enumerate}

Interpretacja geometryczna. \\

Podobnie jak dla zwykłej całki oznaczonej, jeżeli $f \geq 0$ na $(a,b]$ lub $[a,b)$ to całka niewłaściwa 2 rodzaju
$ \int\limits_{a}^{b} f(x) \,dx $ daje pole obszaru ograniczonego osią X, wykresem $f$ oraz prostymi $x=a$ oraz $x=b$.

Najczęściej definiujemy tego typu całkę w przypadku gdy $f$ ma asymptotę pionową $x=a$ lub $x=b$. Wtedy ten obszar
nie jest ograniczony z góry bądź z dołu. \\

Na przykład

$$ \int\limits_{0}^{1} \frac{1}{\sqrt{x}} \,dx = \lim_{T \to 0^+} \int\limits_{T}^{1} \frac{1}{\sqrt{x}} \,dx
\lim_{T \to 0^+} [2\sqrt{x}]_T^1 = \lim_{T \to 0^+} (2 - 2\sqrt{T}) = 2 $$

Całka jest zbieżna do 2. \\

\underline{Wersja całki obustronnej} \\

Ustalamy liczby $ a,b,c \in \mathbb{R}, \ a < c < b $. Niech $f$ będzie funkcją całkowalną na każdym przedziale postaci
$[a,T]$, $T < c$, oraz $ [T, b] $, $ T > c $. Definiujemy całkę niewłaściwą 2 rodzaju z $f$ na zbiorze $[a, c)\cup(c, b] $
jako sumę dwóch całek niewłaściwych. tzn.

$$ \int\limits_{a}^{b} f(x) \,dx = \int\limits_{a}^{c} f(x) \,dx + \int\limits_{c}^{b} f(x) \,dx $$

przy czym gdy przynajmniej jedna z całek z prawej strony nie istnieje lub zachodzi przypadek $ \infty - \infty $ to
$ \int\limits_{a}^{b} f(x) \,dx $ jest rozbieżna, a w pozostałych przypadkach całka ma wartość wynikającą z arytmetyki granic. \\

Najczęściej takie całki pojawiają się, gdy $f$ ma asymptotę w $x = c$. \\

\textbf{Twierdzenie}

Istnieją podstawienia, które każdą całkę niewłaściwą 2 rodzaju sprowadzają do przypadku całki niewłaściwej 1 rodzaju.

W szczególności

\begin{itemize}
    \item dla całki $(a,b]$ możemy wziąć $ t = \frac{1}{x - a} $ co daje $ x = a + \frac{1}{t} $ oraz
    $$ \int\limits_{a}^{b} f(x) \,dx = \int\limits_{C}^{\infty} \frac{1}{t^2} f \left(a + \frac{1}{t} \right) dt \quad
    \textrm{, gdzie} \quad C = \frac{1}{b - a} $$

    \item dla całki na $[a,b)$ możemy wziąć $t = \frac{1}{b - x}$ co daje $ t = b - \frac{1}{t} $ oraz
    $$ \int\limits_{a}^{b} f(x) \,dx = \int\limits_{C}^{\infty} \frac{1}{t^2} f \left(b - \frac{1}{t} \right) dt \quad
    \textrm{, gdzie} \quad C = \frac{1}{b - a} $$ \\
\end{itemize}

Na przykład dla $ p > 0 $ biorąc $ t = \frac{1}{x} $ mamy 

$$ \int\limits_{0}^{b} \frac{1}{x^p} \,dx = \int\limits_{\frac{1}{b}}^{\infty} \frac{1}{t^2} \cdot \frac{1}{ \left( \frac{1}{t} \right)^p } dt
= \int\limits_{\frac{1}{b}}^{\infty} \frac{1}{t^{2 - p}} \,dt $$

Podstawienie to oznacza też, że mamy analogiczne kryteria zbieżności dla całek 2 rodzaju - porównawcze i ilorazowe, przy
czym dla kryterium ilorazowego liczymy granicę ilorazu funkcji w odpowiednim końcu zadanego przedziału. \\

Na koniec, wartość główna całki $ \int\limits_{a}^{b} f(x) \,dx $ na $[a,c)\cup(c,b]$ to wielkość

$$ \textrm{P.V.} \int\limits_{a}^{b} f(x) \,dx = \lim_{T \to 0^+} 
\left( \int\limits_{a}^{c - T} f(x) \,dx + \int\limits_{c + T}^{b} f(x) \,dx \right) $$

o ile powyższa granica istnieje.

Oznacza to, że odpowiednie końce przedziałów całkowania są w jednakowej odległości od c i zbiegają do c.

\subsection*{Zbieżność bezwględna całek niewłaściwych}

Definicja. Całka $ \int\limits_{a}^{\infty} f(x) \,dx $ jest \underline{zbieżna bezwględnie}, gdy zbieżna jest całka
$ \int\limits_{a}^{\infty} |f(x)| \,dx $.

Analogiczne definicje mamy dla pozostałych całek 1 rodzaju oraz dla całek 2 rodzaju. \\

Uwagi

\begin{itemize}
    \item Gdy $f$ jest nieujemna to mamy $ \int\limits_{a}^{\infty} f(x) \,dx = \int\limits_{a}^{\infty} |f(x)| \,dx $
    i definicja nie wnosi nic nowego. Sytuacja się zmienia, gdy są przedziały na którym $f$ ma różne znaki.
    
    \item Nierówność $ \left| \int\limits_{a}^{T} f(x) \,dx \right| \leq \int\limits_{a}^{T} |f(x)| \,dx $ daje
    $ \left| \int\limits_{a}^{\infty} f(x) \,dx \right| \leq \int\limits_{a}^{\infty} |f(x)| dx $ ale gdy są przedziały
    na którym $f$ ma różne znaki to równość nie zachodzi.
    Zatem, ogólnie, $ \left| \int\limits_{a}^{\infty} f(x) \,dx \right| $ i $ \int\limits_{a}^{\infty} |f(x)| \,dx $
    \colorbox{yellow}{to nie to samo}. \\
\end{itemize}

\textbf{Twierdzenie}

Jeżeli całka niewłaściwa jest bezwględnie zbieżna to jest zbieżna (w zwykłym sensie).

Transpozycja tego twierdzenia daje warunek równoważny : 

Jeżeli całka $ \int\limits_{a}^{\infty} f(x) \,dx $ nie jest zbieżna to również nie jest zbieżna bezwględnie,

co oznacza $ \int\limits_{a}^{\infty} |f(x)| \,dx = \infty $.

Analogicznie dla pozostałych typów całek niewłaściwych. \\

Twierdzenie odwrotne nie jest prawdziwe. Są całki zbieżne ale nie bezwględnie, np. $ \int\limits_{1}^{\infty} \frac{\sin x}{x} \,dx $.
Takie całki to tzw. całki \underline{zbieżne warunkowo}.

Są więc 3 możliwe sytuacje - 3 rozłączne podzbiory całek niewłaściwych:

\begin{center}
\includegraphics[scale=0.6]{rozbiezneirozbiezne.png}
\end{center}

Przykład 

Całka $ \int\limits_{1}^{\infty} \frac{\sin x}{\sqrt[3]{x^4}} \,dx $ jest zbieżna bezwględnie, bo biorąc
$ \int\limits_{1}^{\infty} \left| \frac{\sin x}{\sqrt[3]{x^4}} \right| \,dx $ i używając kryterium porównawczego mamy

$$ 0 \leq \left| \frac{\sin x}{\sqrt[3]{x^4}} \right| = \frac{|\sin x|}{x^{\frac{4}{3}}} \leq \frac{1}{x^{\frac{4}{3}}} $$

a całka $ \int\limits_{1}^{\infty} \frac{1}{x^{\frac{4}{3}}} \,dx $ jest zbieżna bo $ \frac{4}{3} > 1 $.
Zatem $ \int\limits_{1}^{\infty} \left| \frac{\sin x}{\sqrt[3]{x^4}} \right| \,dx $ jest zbieżna, a stąd
$ \int\limits_{1}^{\infty} \frac{\sin x}{\sqrt[3]{x^4}} \,dx $ też jest zbieżna.


\section{Szeregi liczbowe}

Dany jest ciąg liczbowy $ a_1, a_2, ..., a_n, ... $

Tworzymy jego ciąg sum częściowych :

$$ S_1 = a_1, \ \ S_2 = a_1 + a_2, \ \ S_n = a_1 + a_2 + ... + a_n = \sum\limits_{k = 1}^{n} a_k $$

Jeżeli istnieje granica $ S = \lim_{n \to \infty} S_n $ (skończona lub nieskończona) to oznaczamy ją symbolem 
$ \sum_{k = 1}^{\infty} a_k $. \\

W ogólnym przypadku możemy wziąć ciąg, który zaczyna się od dowolnej liczby całkowitej \linebreak $ n_0 $ : $ a_{n_0}, a_{n_0 + 1}, ..., a_n, ... $
i jego sum częściowych

$$ S_n = a_{n_0}, \ \ S_{n_0 + 1} = a_{n_0} + a_{n_0 + 1}, \ \ S_n = a_{n_0} + a_{n_0 + 1} + ... + a_n = \sum\limits_{k = n_0}^{n} a_k, \ n \geq n_0 $$

$ S = \lim_{n \to \infty} S_n $ jest oznaczana przez $ \sum\limits_{k = n_0}^{\infty} a_k $. \\

Definicja. Dla ustalonego $ n_0 \in \mathbb{Z} $ obiekt $ \sum\limits_{k = n_0}^{\infty} a_k $ nazywamy \underline{szeregiem liczbowym},
a wartość S (gdy istnieje) jego \underline{sumą}, oznaczaną także przez $ \sum\limits_{k = n_0}^{\infty} a_k $. Mamy wtedy

$$ S_n = a_{n_0}, \ \ S_{n_0 + 1} = a_{n_0} + a_{n_0 + 1}. \ \ S_n = a_{n_0} + a_{n_0 + 1} + ... + a_n + ... = 
\sum\limits_{k = n_0}^{\infty} a_k = \lim_{n \to \infty} \sum\limits_{k = n_0}^{n} a_k = \lim_{n \to \infty} S_n $$

gdzie

\begin{itemize}
    \item $ S_n $ to $n$ - ta suma szeregu,
    \item $ a_n $ to $n$ - ty wyraz szeregu. \\
\end{itemize}

Terminologia dotycząca sumy $S$ jest taka, jak dla ciągów. Są 3 przypadki : 

\begin{enumerate}
    \item $S$ jest liczbą. Wtedy dany szereg jest \underline{zbieżny} (do $S$).
    \item $S = \infty$ lub $S = -\infty$. Wtedy dany szereg jest \underline{rozbieżny} (do $\infty$ lub $-\infty$).
    \item $ S = \lim_{n \to \infty} S_n $ nie istnieje. Wtedy dany szereg jest \underline{rozbieżny}. \\
\end{enumerate}

Przykłady

$$ \frac{1}{2^1} + \frac{1}{2^2} + \frac{1}{2^3} + ... + \frac{1}{2^n} + ... = \sum\limits_{n = 1}^{\infty} \frac{1}{2^n} 
\textrm { - szereg zbieżny do 1}$$

$$ \frac{1}{2} + \frac{1}{3} + ... + \frac{1}{n} + ... = \sum\limits_{n = 2}^{\infty} \frac{1}{n} 
\textrm { - szereg rozbieżny do } \infty$$

$$ 1 - 1 + 1 - 1 + 1 - 1 + ... = \sum\limits_{n = 0}^{\infty} (-1)^n \textrm{ - szereg rozbieżny} $$

Uwaga. Każdy szereg zaczynający się od indeksu $ n_0 \in \mathbb{Z} $ można przekształcić tak, by zaczynał się od indeksu 1.
Wynika to z równości

$$ \sum\limits_{n = n_0}^{\infty} a_n = \sum\limits_{n = 1}^{\infty} a_{n + n_0 - 1} $$

$$ \frac{1}{2} + \frac{1}{3} + ... + \frac{1}{n} + ... = \sum\limits_{n = 2}^{\infty} \frac{1}{n} = \sum\limits_{n = 2}^{\infty} a_n
= \sum\limits_{n = 1}^{\infty} a_{n + 1} = \sum\limits_{n = 1}^{\infty} \frac{1}{n + 1} $$

\subsection*{Obliczanie sum szeregów}

Jest to zadanie trudne, a najczęściej niemożliwe, gdyż trudno jest znaleźć bezpośredni wzór na sumy częściowe $S_n$. \\

Niektóre przypadki szczególne.

\begin{enumerate}
    \item Ciąg geometryczny i szereg geometryczny.
    \begin{itemize}
        \item 
        $ a_n = a_1 \cdot q^{n-1} $, gdzie $q$ jest ilorazem ciągu (czyli $a_{n+1} = a_n \cdot q , \ n \geq 1$).
        
        Wtedy

        $$ S_n = a_1 + a_2 + ... + a_n = a_1 \cdot \frac{1 - q^n}{1 - q}, q \neq 1 \ oraz \ S_n = na_1, \ q = 1 $$

        To oznacza, że dla $ a_1 \neq 0 $,

        \item szereg jest zbieżny dla $ -1 < q < 1 $ i jego suma jest $ S = \frac{a_1}{1 - q} $,
        \item szereg jest rozbieżny do $\infty$ lub $-\infty$ dla $ q \geq 1 $, znak zależy od znaku $a_1$,
        \item szereg jest rozbieżny (suma nie istnieje) dla $ q \leq -1 $ \\
    \end{itemize}

    Stąd np.

    $$ \frac{1}{2^1} + \frac{1}{2^2} + \frac{1}{2^3} + ... + \frac{1}{2^n} + ... = \sum\limits_{n = 1}^{\infty} \frac{1}{2^n} =
    \frac{ \frac{1}{2} }{ 1 - \frac{1}{2} } = 1 \ \textrm{, bo tutaj} \ a_1 = q = \frac{1}{2} $$

    \item Szeregi o wyrazie ogólnym postaci \\
    $ a_n = f(n + 1) - f(n)$ lub $ a_n = f(n) - f(n + 1) $, gdzie $f$ jest pewną funkcją.

    W bardziej ogólnej postaci \\
    \quad $ a_n = f(n + k) - f(n) $ lub $ a_n = f(n) - f(n + k) $, gdzie $ k \in \mathbb{N}^+ $ to tzw. krok. \\
    
    Takie szeregi to tzw. szeregi \underline{teleskopowe} (telescoping series).

    Przykłady

    $$ \sum\limits_{n = 1}^{\infty} \left( \frac{1}{n} - \frac{1}{n + 1} \right) \ \textrm{- tutaj} \ f(x) = \frac{1}{x} $$

    $$ \sum\limits_{n = 1}^{\infty} \left( \sqrt{n + 1} - \sqrt{n} \right) \ \textrm{- tutaj} \ f(x) = \sqrt{x} $$

    $$ \sum\limits_{n = 1}^{\infty} \left( \arctan(n) - \arctan(n+2) \right) \ \textrm{- tutaj} \ f(x) = \arctan x $$

    Dla takich szeregów łatwo wyznacza się wzór na $S_n$. Wyrazy wewnętrzne się upraszczają i zostaje: \\
    suma $k$ pierwszych wartości, \ $f$ suma $k$ ostatnich wartości $f$ (lub na odwrót) \\

\end{enumerate}

Na przykład dla $ \sum\limits_{n = 1}^{\infty} \left( f(n) - f(n + 1)) \right) $ mamy

$$ S_n = f(1) - \textcolor{red}{f(2)} + \textcolor{red}{f(2)} - \textcolor{blue}{f(3)} + \textcolor{blue}{f(3)} - 
\textcolor{green}{f(4)} + ... + \textcolor{purple}{f(n)} - f(n + 1) = f(1) - f(n + 1) $$.

Jeżeli istnieje granica $ G = \lim_{x \to \infty} f(x) $ to mamy

$$ S = \lim_{n \to \infty} S_n = \lim_{n \to \infty} \left( f(1) - f(n + 1) \right) = f(1) - G $$ \\

Przykład. Wyznaczyć sumę $ \sum\limits_{n = 1}^{\infty} \frac{1}{n^2 + n} $

Wyraz ogólny nie ma postaci różnicy więc trzeba ją stworzyć.

Używając rozkładu na ułamki proste dostajemy

$$ \frac{1}{n^2 + n} = \frac{1}{n(n+1)} = \frac{A}{n} + \frac{B}{n+1} = ... = \frac{1}{n} - \frac{1}{n+1} $$

Zatem 

$$ \sum\limits_{n = 1}^{\infty} \frac{1}{n^2 + n} = \sum\limits_{n=1}^{\infty} \left( \frac{1}{n} - \frac{1}{n+1} \right) $$

I to daje

$$ S_n = \left( \frac{1}{1} - \frac{1}{2} \right) + \left( \frac{1}{2} - \frac{1}{3} \right) + \left( \frac{1}{3} - \frac{1}{4} \right)
+ ... + \left( \frac{1}{n} - \frac{1}{n+1} \right) = \frac{1}{1} - \frac{1}{n+1}  $$ 

$$ \lim_{n \to \infty} S_n = 1 = \sum\limits_{n = 1}^{\infty} \frac{1}{n^2 + n} $$

\subsection*{Własności szeregów zbieżnych}

\textbf{Twierdzenie}

Jeżeli szeregi $ \sum\limits_{n = n_0}^{\infty} a_n $ oraz $ \sum\limits_{n = n_0}^{\infty} b_n $ są zbieżne to zbieżne są szeregi
$ \sum\limits_{n = n_0}^{\infty} (a_n + b_n) $ \linebreak oraz $ \sum\limits_{n = n_0}^{\infty} (c \cdot a_n), \ c \in \mathbb{R} $. \\

Ponadto

\begin{itemize}
    \item $ \sum\limits_{n = n_0}^{\infty} (a_n \pm b_n) = \sum\limits_{n = n_0}^{\infty} a_n \pm \sum\limits_{n = n_0}^{\infty} b_n $
    \item $ \sum\limits_{n = n_0}^{\infty} (c \cdot a_n) = c \sum\limits_{n = n_0}^{\infty} a_n $ \\
\end{itemize}

Prawdziwe są także analogiczne twierdzenia prowadzące do arytmetyki granic nieskończonych, gdy nie pojawiają się symbole nieoznaczone.
Na przykład gdy $ \sum\limits_{n = n_0}^{\infty} a_n = \infty $ oraz $ \sum\limits_{n = n_0}^{\infty} b_n = b \in \mathbb{R} $ to 

$$ \sum\limits_{n = n_0}^{\infty} (a_n \pm b_n) = \sum\limits_{n = n_0}^{\infty} a_n \pm \sum\limits_{n = n_0}^{\infty} b_n = \infty $$ \\

Natomiast gdy $ \sum\limits_{n = n_0}^{\infty} a_n = \sum\limits_{n = n_0}^{\infty} b_n = \infty $ to
$ \sum\limits_{n = n_0}^{\infty} (a_n - b_n) $ może być zarówno zbieżny jak i rozbieżny i nie ma sensu równość

$$ \sum\limits_{n = n_0}^{\infty} (a_n - b_n) = \sum\limits_{n = n_0}^{\infty} a_n - \sum\limits_{n = n_0}^{\infty} b_n $$ \\

\textbf{Twierdzenie}

Zmiana wartości $n_0$ nie wpływa na zbieżność/rozbieżność szeregu $ \sum\limits_{n = n_0}^{\infty} a_n $.

Może mieć wpływ na wartość jego sumy. \\

Stąd wynika np., że szeregi $ \sum\limits_{n = 1}^{\infty} a_n $ i $ \sum\limits_{n = 100}^{\infty} a_n $ są albo oba zbieżne
albo oba rozbieżne do $ \infty $ lub $-\infty$ albo oba rozbieżne.

To też oznacza, że na podstawie kilku pierwszych wyrazów ciągu/szeregu

\colorbox{yellow}{NIC NIE MOŻNA POWIEDZIEĆ} o jego zbieżności \\

\colorbox{red}{Popularny błąd}

"Liczymy wartości $ a_1, a_2, a_3, a_4, a_5$ . Wychodzi ciąg malejący i dodatni.

\textcolor{red}{Zatem szereg jest zbieżny}". \textbf{GAME OVER...} \\

\textbf{Twierdzenie}

Dla ustalonego $ n_0 \in \mathbb{N}^+ $ i $ p \in \mathbb{R} $ szereg $ \sum\limits_{n = n_0}^{\infty} \frac{1}{n^p} $
jest zbieżny dla $ p > 1 $ i rozbieżny do $\infty$ dla $p \leq 1$.  \\

W przypadku kiedy sumy szeregu nie da się wyznaczyć w sposób dokładny można to zrobić w sposób przybliżony, pod warunkiem, że wiemy,
że szereg jest zbiezny. \\

Kryteria zbieżności to twierdzenia opisujące warunki dostateczne zbieżności lub rozbieżności danej klasy szeregów. Najczęściej mają postać
implikacji ale \textbf{NIE} równoważności. \\

Oznacza to zwykle własności postaci

\quad warunek zachodzi $ \Rightarrow $ szereg jest zbieżny/rozbieżny,

\quad warunek nie zachodzi $\Rightarrow$ nic nie wiemy o zbieżności/rozbieżności szeregu \\

\subsection*{Popularne kryteria zbieżności szeregów}

0. Warunek konieczny zbieżności szeregów \\ 

\textbf{Twierdzenie}

Jeżeli szereg $ \sum\limits_{n = n_0}^{\infty} a_n $ jest zbieżny to $ \lim_{n \to \infty} a_n = 0 $. \\

Dowód 

Dla $ n \geq n_0 + 1 $ mamy $ S_n = a_{n_0} + a_{n_0 + 1} + ... + a_{n - 1} + a_n $ oraz 
$ S_{n - 1} = a_{n_0} + a_{n_0 + 1} + ... + a_{n - 1} $,

Stąd

$$ S_n - S_{n - 1} = a_n $$ \\

Jeżeli szereg $ \sum\limits_{n = n_0}^{\infty} a_n $ jest zbieżny to $ \lim_{n \to \infty} S_n = \lim_{n \to \infty} S_{n - 1} = S \in \mathbb{R} $.
To daje

$$ \lim_{n \to \infty} a_n = \lim_{n \to \infty} (S_n - S_{n - 1}) = \lim_{n \to \infty} S_n - \lim_{n \to \infty} S_{n - 1} = S - S = 0 $$ \\

Transpozycja tego twierdzenia daje warunek równoważny do zastosowania praktycznego :

Jeżeli $ \lim_{n \to \infty} a_n \neq 0 $ to szereg $ \sum\limits_{n = n_0}^{\infty} a_n $ nie jest zbiezny przy czym

\begin{itemize}
    \item gdy $ \lim_{n \to \infty} a_n > 0 $ to $ \sum\limits_{n = n_0}^{\infty} a_n = \infty $
    \item gdy $ \lim_{n \to \infty} a_n < 0 $ to $ \sum\limits_{n = n_0}^{\infty} a_n = -\infty $ 
\end{itemize}

\colorbox{yellow}{Uwaga. To jest tylko implikacja!}

Jeżeli $ \lim_{n \to \infty} a_n = 0 $ to jeszcze \colorbox{yellow}{NIC NIE WIEMY} o szeregu.

Na przykład szeregi $ \sum\limits_{n = n_0}^{\infty} \frac{1}{n^p} $ mają $ \lim_{n \to \infty} \frac{1}{n^p} = 0 $
dla wszystkich $ p > 0 $ ale niektóre z tych szeregów są zbieżne, a niektóre rozbieżne. \\

\colorbox{red}{Popularny błąd}

"$\lim_{n \to \infty} a_n = 0 $ \textcolor{red}{zatem szereg jest zbieżny}". \textbf{GAME OVER...} \\

\textbf{DZIURA W SKRYPCIE} \\

Podobnie dla szeregów o wyrazach niedodatnich $ \sum\limits_{n = n_0}^{\infty} a_n, \ a_n \leq 0 $, suma zawsze istnieje
i rozbieżność oznacza rozbieżność do $-\infty$. \\

\pagebreak

Dla szeregów o wyrazach nieujemnych mamy dwa kolejne kryteria zbieżności.

\begin{enumerate}
    \item Kryterium porównawcze
    \item Kryterium ilorazowe \\
\end{enumerate} 

\subsection*{Twierdzenie (kryterium porównawcze)}

Dane są dwa szeregi $ \sum\limits_{n = n_0}^{\infty} a_n $ oraz $ \sum\limits_{n = n_0}^{\infty} b_n $. Wtedy zachodzą nastepujące
własności.

\begin{enumerate}
    \item (Przypadek zbieżności) Gdy $ \forall n \geq k \geq n_0 \quad 0 \leq a_n \leq b_n $ i $ \sum\limits_{n = n_0}^{\infty} b_n $
    jest zbieżny to $ \sum\limits_{n = n_0}^{\infty} a_n $ też jest zbieżny. Ponadto
    $ 0 \leq \sum\limits_{n = n_0}^{\infty} a_n \leq \sum\limits_{n = n_0}^{\infty} b_n $
    
    \item (Przypadek rozbieżności) Gdy $ \forall n \geq k \geq n_0 \quad 0 \leq b_n \leq a_n $ i $ \sum\limits_{n = n_0}^{\infty} b_n $
    jest rozbieżny to $ \sum\limits_{n = n_0}^{\infty} a_n $ też jest rozbieżny. Ponadto
    $ \sum\limits_{n = n_0}^{\infty} a_n = \sum\limits_{n = n_0}^{\infty} b_n = \infty $

    \item (Przypadek wątpliwy). Gdy $ \forall n \geq k \geq n_0 \quad 0 \leq a_n \leq b_n $ ale $ \sum\limits_{n = n_0}^{\infty} b_n $ jest
    rozbieżny to \textbf{NIC NIE WIEMY} o zbieżności $ \sum\limits_{n = n_0}^{\infty} a_n $

    \item (Przypadek wątpliwy). Gdy $ \forall n \geq k \geq n_0 \quad 0 \leq b_n \leq a_n $ ale $ \sum\limits_{n = n_0}^{\infty} b_n $ jest
    zbieżny to \textbf{NIC NIE WIEMY} o zbieżności $ \sum\limits_{n = n_0}^{\infty} a_n $ \\
\end{enumerate}

Uwagi

\begin{itemize}
    \item $ \sum\limits_{n = n_0}^{\infty} a_n $ jest szeregiem z zadania, $ \sum\limits_{n = n_0}^{\infty} b_n $ tworzymy sami
    \item Porównujemy najczęściej z szeregiem geometrycznym $ \sum\limits_{n = n_0}^{\infty} q^n $ lub z szeregami
    $ \sum\limits_{n = n_0}^{\infty} \frac{1}{n^p} $. Wtedy $a_n$ często ma postać ułamków i możemy spróbować wziąć $b_n$ jako

    \quad \colorbox{yellow}{C $\cdot$ iloraz najwyższych potęg z licznika i mianownika $a_n$} 

    \item Trzeba uważać aby nierówność między $a_n$ i $b_n$ była prawdziwa i nie zapomnieć o dolnym ograniczeniu (0). Ma być
    tak jak w \underline{twierdzeniu o trzech ciągach}

    \item Kryterium nie zawsze jest wygodne w użyciu i trzeba uważać, by nie dostać przypadku wątpliwego, bo wtedy \textbf{trzeba zaczynać od nowa}
    
    \item Warto sprawdzić opisany wyżej iloraz najwyższych potęg i na tej podstawie przewidzieć czy chcemy udowodnić zbieżność
    czy rozbieżność. To pomaga skonstruować odpowiednią nierówność między $a_n$ i $b_n$.
\end{itemize}

\colorbox{red}{Popularny błąd : } odpowiedź na podstawie przypadku wątpliwego

Na przykład dla szeregu $ \sum\limits_{n = 1}^{\infty} \frac{1}{n + \sqrt{n}} : $

"Mamy $ 0 \leq \frac{1}{n + \sqrt{n}} \leq \frac{1}{n} $ i szereg $ \sum\limits_{n = 1}^{\infty} \frac{1}{n} $
jest rozbieżny \textcolor{red}{zatem szereg $\sum\limits_{n = 1}^{\infty} \frac{1}{n + \sqrt{n}}$ jest rozbieżny}". \\

\textbf{GAME OVER...} To jest przypadek nr 3 (wątpliwy) \\

Przykład

$$ \sum\limits_{n = 4}^{\infty} \frac{2n - 3}{n^3 - 1} $$

Przewidywanie zbieżności/rozbieżności:

Iloraz najwyższych potęg licznika i mianownika to $ \frac{n}{n^3} = \frac{1}{n^2} $, a szereg $ \sum\limits_{n = 4}^{\infty} \frac{1}{n^2} $
jest zbieżny, bo $ 2 > 1 $. Zatem chcemy udowodnić zbieżność (przypadek 1).

Potrzebujemy więc $ \sum\limits_{n = 4}^{\infty} b_n $ i nierówności $ 0 \leq \frac{2n - 3}{n^3 - 1} \leq b_n $.

Chcemy zwiększyć wyrażenie $ \frac{2n - 3}{n^3 - 1} $ ale tak, by \colorbox{yellow}{zostały najwyższe potęgi}.

Można zwiększyć \underline{licznik} oraz \underline{zmniejszyć mianownik}.

Zwiększamy licznik poprzez wyrzucenie 3.

Zmniejszamy mianownik poprzez zastąpienie 1 czymś większym : wyrażeniem z najwyższą potęgą. Nie można jednak wziąć całego $n^3$
, bo będzie 0 w mianowniku.

Wygrywa wzięcie $ C \cdot n^3 $ np. $ \frac{1}{2} n^3 $, bo dla $ n \geq 4 $ mamy $ \frac{1}{2} n^3 \geq 1 $.

To wszystko daje dla $ n \geq 4 $

$$ 0 \leq \frac{2n - 3}{n^3 - 1} \leq \frac{2n}{n^3 - \frac{1}{2} n^3} $$

Czyli

$$ b_n = \frac{2n}{n^3 - \frac{1}{2}n^3} = 4 \cdot \frac{1}{n^2} $$



\end{document}