\chapter{Całki niewłaściwe I rodzaju}

\begin{tw}{Definicja}
    Ustalamy liczbę $a \in \mathbb{R}$. Niech $f$ będzie funkcją całkowalną na każdym przedziale \linebreak w postaci $[a, T]$
    gdzie $T > a$. Definiujemy \underline{całkę niewłaściwą pierwszego rodzaju} z $f$ na półprostej $[a, \infty]$ jako

    \[ \int\limits_{a}^{\infty} f(x) \,dx = \lim_{T \to \infty} \int\limits_{a}^{T} f(x) \,dx \ \text{,  gdy granica po prawej stronie istnieje} \]

    Analogicznie, gdy $f$ jest całkowalna na każdym przedziale postaci $[T, b]$, gdzie $T < b$. Definiujemy całkę niewłaściwą
    pierwszego rodzaju z $f$ na półprostej [$-\infty$, b] jako

    \[ \int\limits_{-\infty}^{b} f(x) \,dx = \lim_{T \to -\infty} \int\limits_{T}^{b} f(x) \,dx \ \text{,  gdy granica po prawej stronie istnieje} \]

\end{tw}

Terminologia dotycząca takich całek jest taka, jak dla ciągów. Są 3 przypadki :

\begin{enumerate}
    \item Granica z prawej strony jest liczbą. Wtedy mówimy, że całka jest \underline{zbieżna}.
    \item Granica z prawej strony jest równa $\infty$ lub $-\infty$. Wtedy mówimy, że całka jest \underline{rozbieżna} (odpowiednio do $\infty$ lub $-\infty$).
    \item Granica z prawej strony nie istnieje. Wtedy mówimy, że całka jest \underline{rozbieżna}.
\end{enumerate}

Analogicznie dla $ \int\limits_{\infty}^{b} f(x)\,dx $

\begin{przyklad}
$$ \int\limits_{0}^{\infty} \sin x \,dx = \lim_{T \to \infty} \int\limits_{0}^{T} \sin x \,dx = 
\lim_{T \to \infty} [-\cos x]_0^T = \lim_{T \to \infty} (-\cos T - (- \cos 0)) = \lim_{T \to \infty} (1 - \cos T) $$

Granica ta nie istnieje więc całka jest rozbieżna. 
\end{przyklad}

\begin{przyklad}
$$ \int\limits_{-\infty}^{0} 2^x \,dx = \lim_{T \to -\infty} \int\limits_{T}^{0} 2^x \,dx = 
\lim_{T \to -\infty} \left[ \frac{2^x}{\ln 2} \right]_T^0 = \lim_{T \to -\infty} 
\left( \frac{1}{\ln 2} - \frac{2^T}{\ln 2} \right) = \frac{1}{\ln 2} $$

Całka jest zbieżna do $ \frac{1}{\ln 2} $.

Pozostaje przypadek $ p = 1 $. Wtedy

$$ \int \frac{1}{x} \,dx = \ln |x| + C, \ \
\int\limits_{a}^{T} \frac{1}{x} \,dx = [\ln |x|]_a^T = \ln |T| - \ln |a|, \ \
\int\limits_{a}^{\infty} \frac{1}{x} \,dx = \lim_{T \to \infty} (\ln |T| - \ln |a|) = \infty $$

Udowodniliśmy zatem ważny wynik
\end{przyklad}

\begin{tw}{Twierdzenie} 

Gdy $ a > 0 $ to całka $ \int\limits_{a}^{\infty} \frac{1}{x^p} \,dx $
jest skończona dla $ p > 1 $ oraz nieskończona dla $ p \leq 1 $.

\end{tw}

Podobnie można łatwo pokazać poniższy wynik \\

\begin{tw}{Twierdzenie}

Gdy $ a \in \mathbb{R} $ i $ A > 0 $ to całka $ \int\limits_{a}^{\infty} A^x \,dx $
jest skończona dla $ 0 < A < 1 $ oraz nieskończona \linebreak dla $ A \geq 1 $ \\

\end{tw}

Gdy $ \int f(x) \,dx = F(x) + C $ \ to

$$ \int\limits_{-\infty}^{\infty} f(x) \, dx = \lim_{T \to \infty} F(T) - \lim_{S \to \infty} F(S) $$

przy czym przynajmniej jedna z granic z prawej strony nie istnieje lub zachodzi przypadek 
$ \infty - \infty $ to $ \int\limits_{-\infty}^{\infty} f(x) \,dx $ jest rozbieżna, a w pozostałych
przypadkach całka ma wartość wynikającą z arytmetyki granic. \\

W przypadku kiedy całki nie da się obliczyć w sposób dokładny można to zrobić w sposób przybliżony, pod warunkiem
, że wiemy, że jest zbieżna.

Kryteria zbieżności to twierdzenia opisujące warunki dostateczne zbieżności lub rozbieżności danej klasy
całek. Najczęściej mają postać implikacji ale NIE równoważności. \\

Oznacza to zwykle własności postaci

\quad warunek zachodzi $ \Rightarrow $ całka jest zbieżna/rozbieżna

\quad warunek nie zachodzi $ \Rightarrow $ nic nie wiemy o zbieżności/rozbieżności całki

\subsection{Popularne kryteria zbieżności całek z $\infty$}

Warunek konieczny zbieżności całki

\begin{tw}{Twierdzenie}
Jeżeli całka $ \int\limits_{a}^{\infty} f(x) \,dx $ jest zbieżna to 
$ \lim\limits_{x \to \infty} f(x) $ jest równa 0 lub nie istnieje.
\end{tw}

Transpozycja twierdzenia daje następujący wynik:

\begin{tw}{Twierdzenie}
Jeżeli $ \lim\limits_{x \to \infty} f(x) $ istnieje i jest różna od 0 to całka 
$ \int\limits_{a}^{\infty} f(x) \, dx $ nie jest zbieżna, przy czym

\begin{itemize}
    \item gdy $ \lim\limits_{x \to \infty} f(x) > 0 $ to $ \int\limits_{a}^{\infty} f(x) \,dx = \infty $,
    \item gdy $ \lim\limits_{x \to \infty} f(x) < 0 $ to $ \int\limits_{a}^{\infty} f(x) \,dx = -\infty $,
\end{itemize}

\end{tw}

\textbf{Uwaga. Warunek konieczny to tylko implikacja!}

Jeżeli $ \lim_{x \to \infty} f(x) $ jest równa 0 lub nie istnieje to jeszcze \textbf{NIC NIE WIEMY} o całce,

Na przykład całki $ \int\limits_{a}^{\infty} \frac{1}{x^p} \,dx, \ a > 0 $, mają
$ \lim_{x \to \infty} \frac{1}{x^p} = 0 $ dla wszystkich $ p > 0 $ ale niektóre z tych całek są zbieżne,
a niektóre rozbieżne \\

\subsection*{Ważna klasa całek - całki z funkcji nieujemnych}

$$ \int\limits_{a}^{\infty} f(x) \,dx, \ f \geq 0 $$

Wtedy $ \int\limits_{a}^{T} f(x) \, dx = F(T) - F(a) $ jest funkcją niemalejącą zmiennej T zatem całka
$ \int\limits_{a}^{\infty} f(x) \, dx = \lim_{T \to \infty} \int\limits_{a}^{T} f(x) \,dx $
zawsze istnieje. Może być to liczba lub $\infty$.

Zatem brak zbieżności takich całek oznacza rozbieżność do $\infty$. \\

Dla całek z funkcji nieujemnych mamy dwa kolejne kryteria zbieżności.

\begin{enumerate}
    \item Kryterium porównawcze
    \item Kryterium ilorazowe
\end{enumerate}


\subsection{Twierdzenie(kryterium porównawcze)}

Dane są dwie całki $ \int\limits_{a}^{\infty} f(x) \,dx $ oraz
$ \int\limits_{a}^{\infty} g(x) \, dx $. Wtedy zachodzą następujące własności

\begin{enumerate}
    \item (Przypadek zbieżności). Gdy $ \forall x \geq x_0 \geq a \ \ 0 \leq f(x) \leq g(x) $ i $ \int\limits_{a}^{\infty} g(x) \,dx $
    jest zbieżna to $ \int\limits_{a}^{\infty} f(x) \,dx $ też jest zbieżna. Ponadto \
    $ 0 \leq \int\limits_{a}^{\infty} f(x) \,dx \leq \int\limits_{a}^{\infty} g(x) \,dx $
    
    \item (Przypadek rozbieżności) Gdy $ \forall x \geq x_0 \geq a \ \ 0 \leq g(x) \leq f(x) $ i $ \int\limits_{a}^{\infty} g(x) \,dx $
    jest rozbieżna (więc równa $\infty$) to $ \int\limits_{a}^{\infty} f(x) \,dx $ też jest rozbieżna (do $\infty$).
    
    \item \textbf{(Przypadek wątpliwy)} Gdy $ \forall x \geq x_0 \geq a \ \ 0 \leq f(x) \leq g(x) $ ale $ \int\limits_{a}^{\infty} g(x) \,dx $
    jest rozbieżna to \textbf{NIC NIE WIEMY} o zbieżności $ \int\limits_{a}^{\infty} f(x) \,dx $.
    
    \item \textbf{(Przypadek wątpliwy)} Gdy $ \forall x \geq x_0 \geq a \ \ 0 \leq g(x) \leq f(x) $ ale $ \int\limits_{a}^{\infty} g(x) \,dx $
    jest zbieżna to \textbf{NIC NIE WIEMY} o zbieżności $ \int\limits_{a}^{\infty} f(x) \,dx $.
\end{enumerate}

Uwagi:

\begin{itemize}
    \item $ \int\limits_{a}^{\infty} f(x) \,dx $ jest całką z zadania, $ \int\limits_{a}^{\infty} g(x) \,dx $ tworzymy sami.
    \item Porównujemy najczęściej z całkami $ \int\limits_{a}^{\infty} A^x \,dx $ lub 
    $ \int\limits_{a}^{\infty} \frac{1}{x^p} \,dx $. Wtedy $f$ często ma postać ułamków i możemy spróbować
    wziąć $g$ jako :

    \quad C - iloraz najwyższych potęg z licznika i mianownika $f$

    \item Trzeba uważać aby nierówność między $f$ i $g$ była prawdziwa i nie zapomnieć przypadku wątpliwego, bo wtedy
    \textbf{trzeba zaczynać od nowa}.

    \item Warto sprawdzić opisany wyżej iloraz najwyższych potęg i na tej podstawie przewidzieć czy chcemy
    udowodnić zbieżność czy rozbieżność. To pomaga skonstruować odpowiednią nierówność między $f$ i $g$.
\end{itemize}

\begin{blad}{Popularny błąd - odpowiedź na podstawie przypadku wątpliwego}

Na przykład dla całki $ \int\limits_1^\infty \frac{1}{x + \sqrt{x}} \,dx $ : 

"Mamy \ $ 0 \leq \frac{1}{x + \sqrt{x}} \leq \frac{1}{x} $ \ i całka \ $ \int\limits_1^\infty \frac{1}{x} \,dx $
jest rozbieżna \textcolor{red}{zatem całka $ \int\limits_{1}^{\infty} \frac{1}{x + \sqrt{x}} \,dx $ jest rozbieżna.}"

GAME OVER... To jest przypadek nr 3 (wątpliwy) \\

\end{blad}

\begin{przyklad}

$$ \int\limits_4^\infty \frac{2x - 3}{x^3 - 1} \,dx $$

Przewidywanie zbieżności/rozbieżności

Najwyższe potęgi sugerują, że mając

$$ \frac{x}{x^3} = \frac{1}{x^2}, \quad \textrm{ a} \quad \int\limits_4^\infty \frac{1}{x^2} \,dx < \infty, \quad \textrm{bo} \quad 2 > 1 $$

Dowodzimy zbieżność. Trzeba mieć
$$ 0 \leq \frac{2x - 3}{x^3 - 1} \leq g(x) = C \cdot \frac{x}{x^3} $$

Jak w twierdzeniu o 3 ciągach

$$ 0 \leq \frac{2x}{x^3 - \frac{1}{2}x^3} = 4 \cdot \frac{x}{x^3} = 4 \cdot \frac{1}{x^2} $$

$$ \int\limits_4^\infty \frac{4}{x^2} \,dx = 4 \int\limits_4^\infty \frac{1}{x^2} \,dx < \infty
\quad \left(\frac{1}{2}x^3 > 1 \ \mathrm{dla} \ x \geq 4 \right) $$
\end{przyklad}


\subsection{Twierdzenie(kryterium ilorazowe)}

Dane są dwie całki $ \int\limits_a^\infty f(x) \,dx $ oraz $ \int\limits_{a}^{\infty} g(x) \,dx $. Ponadto

$$ \forall x \geq x_0 \geq a \quad f(x), g(x) > 0 $$

Jeżeli istnieje granica $ \lim_{x \to \infty} \frac{f(x)}{g(x)} $ i jest \underline{liczbą dodatnią} to wtedy obie całki
są zbieżne albo obie rozbieżne do $\infty$. \\

Uwagi
\begin{itemize}
    \item Funkcję $g$ tworzymy podobnie jak dla kryterium porównawczego
    \item Nie ma problemu z nierównościami :) ale za to trzeba umieć liczyć granice
    \item Granica nie może być ani 0 ani $\infty$: $ \lim_{x \to \infty} \frac{f(x)}{g(x)} \in (0, \infty) $
    \item Rozwiązanie \textbf{musi zawierać wniosek} "granica ilorazu jest liczbą dodatnią więc obie całki
    są zbieżne lub obie rozbieżne" - bez tego będzie niepełne.
    \item Kryterium zwykle jest wygodniejsze niż porównawcze ale są przykłady, które "idą" z porównawczego ale nie z
    ilorazowego, bo granica ilorazu nie istnieje

    Np. $ \int\limits_{1}^{\infty} \frac{2 + \sin x}{x} \,dx $
\end{itemize}

\begin{przyklad}

Poprzedni przykład raz jeszcze 

$$ \int\limits_4^\infty \frac{2x - 3}{x^3 - 1} \,dx $$

$$ f(x) = \frac{2x - 3}{x^3 - 1}, \quad x \geq 4 $$

$$ g(x) = \frac{x}{x^3} = \frac{1}{x^2} > 0 $$

$$ \lim_{x \to \infty} = \frac{f(x)}{g(x)} = \lim_{x \to \infty} \frac{x^2(2x - 3)}{x^3 - 1} = 2 $$

Obie całki zbieżne lub obie rozbieżne do $\infty$
\end{przyklad}

Przykłady o postaci funkcji złożonej $ \int\limits_{a}^{\infty} f(g(x)) \,dx $
gdzie $ \lim_{x \to \infty} g(x) = 0^+ $ oraz $ \lim_{x \to 0^+} f(x) = 0^+ $

Nową całką jest całka z funkcji wewnętrznej $ \int\limits_{a}^{\infty} g(x) \,dx $ \\

Liczymy granicę

$$ \lim_{x \to \infty} \frac{f(g(x))}{g(x)} = \lim_{t = g(x) \to 0^+} \frac{f(t)}{t} \left[ \frac{0}{0} \right] $$ \\

przy użyciu granic podstawowych lub reguły de l'Hospitala. \\

\begin{przyklad}

$$ \int\limits_{1}^{\infty} \left( 2^{\frac{1}{\sqrt{x}}} - 1 \right) \,dx $$

$$ g(x) = \frac{1}{\sqrt{x}} > 0 $$

$$ f(x) = 2^x - 1 > 0 $$

$$ \lim_{x \to \infty} \frac{2^{\frac{1}{\sqrt{x}}} - 1}{\frac{1}{\sqrt{x}}} = \lim_{t \to 0^+} \frac{2^t - 1}{t}
\left[ \frac{0}{0} \right] = \ln 2 \in (0, \infty) $$

Obie całki zbieżne lub obie rozbieżne

$$ \int\limits_1^\infty \frac{1}{\sqrt{x}} \,dx = \int\limits_1^\infty \frac{1}{x^{\frac{1}{2}}} \,dx = \infty\ \quad
\textrm{bo} \quad \frac{1}{2} \leq 1 $$
\end{przyklad}

\subsection{Wartość główna całki niewłaściwej I rodzaju}

Całka $ \int\limits_{-\infty}^{\infty} x \,dx $ jest rozbieżna, gdyż jako suma całek prowadzi do symbolu $ \infty - \infty $:

$$ \int\limits_{-\infty}^{\infty} x \,dx = \int\limits_{-\infty}^{0} x \,dx + \int\limits_{0}^{\infty} x \,dx = -\infty + \infty $$

Intuicyjnie oczekwialibyśmy jednak, że jest ona równa 0 - funkcja podcałkowa jest nieparzysta czyli mamy "tyle funkcji
na + co na -", a więc wszystko powinno się wzajemnie zrównoważyć.

Aby taka całka miała sens trzeba nieco zmodyfikować jej definicję i wprowadzić pojęcie wartości głównej całki niewłaściwej (obustronnej).

\begin{tw}{Definicja}
Wartość główna całki $ \int\limits_{-\infty}^{\infty} f(x) \,dx $ to wielkość

$$ \textrm{P.V.} \int\limits_{-\infty}^{\infty} f(x) \,dx = \lim_{T \to \infty} \int\limits_{-T}^{T} f(x) \,dx $$

o ile powyższa granica istnieje.
\end{tw}

Oznacza to, że przybliżamy całkę po $\mathbb{R}$ całkami po przedziale symetrycznym względem 0.

P.V. jest skrótem od angielskiego "Principal Value". \\

\begin{przyklad}

$$ \textrm{P.V.} \int\limits_{-\infty}^{\infty} x \,dx = \lim_{T \to \infty} \int\limits_{-T}^{T} x\,dx
= \lim_{T \to \infty} 0 = 0 $$

Zauważmy, że gdy $ \int f(x) \,dx = F(x) + C $ to

$$ \textrm{P.V.} \int\limits_{-\infty}^{\infty} f(x) \,dx = \lim_{T \to \infty} \int\limits_{-T}^{T} f(x) \,dx
= \lim_{T \to \infty} (F(T) - F(-T)) $$

Jeżeli teraz ma sens wyrażenie $ \lim_{T \to \infty} F(T ) - \lim_{T \to \infty} F(-T) $ to biorąc $ S = -T \to -\infty $ dostajemy

$$ \textrm{P.V.} \int\limits_{-\infty}^{\infty} f(x) \,dx = \lim_{T \to \infty} (F(T) - F(-T)) = 
\lim_{T \to \infty} F(T) - \lim_{T \to \infty} F(-T) = $$ $$ =  \lim_{T \to \infty} F(T) - \lim_{S \to -\infty} F(S)
= \int\limits_{-\infty}^{\infty} f(x) \,dx $$
\end{przyklad}

Udowodniliśmy zatem poniższe twierdzenie. 

\begin{tw}{Twierdzenie}
Jeżeli całka $ \int\limits_{-\infty}^{\infty} f(x) \,dx $ istnieje w zwykłym sensie (jako suma odpowiednich całek jednostronnych
jest liczbą lub jedną z nieskończoności) to również jej wartość główna istnieje i jest równa tej całce.
\end{tw}

Natomiast może się zdarzyć, że wartość główna całki istnieje ale sama całka jest rozbieżna (był przykład).

W szczególności gdy funkcja jest na $\mathbb{R}$ ciągła i nieparzysta to wartość główna całki z tej funkcji jest zawsze
0 niezależnie od zbieżności samej całki.